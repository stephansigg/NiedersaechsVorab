\documentclass[12pt]{article}
\usepackage[utf8]{inputenc}
%\usepackage[ansinew]{inputenc}
\usepackage{a4wide}
\usepackage{fancyhdr}
\usepackage{eurosym}
\usepackage{graphicx}
\usepackage{xcolor,soul}
%\usepackage[pdftex]{graphicx}
%\selectlanguage{german}
%\setlength{\topmargin}{-0.5cm}
\setlength{\textheight}{23cm}
\setlength{\oddsidemargin}{-.4cm}
\setlength{\topmargin}{-1cm}
\setlength{\textwidth}{16.7cm}
\pagestyle{fancy}
\renewcommand{\familydefault}{\sfdefault}
% Deine Adresse:
\newcommand{\Name}{Stephan Sigg}
\newcommand{\Strasse}{Burg 4}
\newcommand{\Ort}{Braunschweig}
\newcommand{\Plz}{38124}
\newcommand{\Telefon}{+49~5341~4022052}
\newcommand{\Email}{stephan.sigg@udo.edu}
\newcommand{\Country}{Germany}

\lhead{\tiny \Name, \Strasse, \Plz\ \Ort, \Country. Phone: \Telefon, Email: \Email}
\cfoot{\footnotesize page \arabic{page} of \pageref{MaxSeitenzahl}}


\definecolor{lightgray}{rgb}{0.83, 0.83, 0.83}


%\lfoot{Neu}
%\pagestyle{myheadings}
%\markboth{\scriptsize \hfill Stephan Sigg \hfill $\bullet$ \hfill Hessische Str. 165 \hfill $\bullet$ \hfill
%&44339 Dortmund \hfill $\bullet$ \hfill Tel. 0231/3997729\hfill}
%{\scriptsize \hfill Stephan Sigg\hfill $\bullet$ \hfill Hessische Str. 165 \hfill $\bullet$ \hfill 44339
%Dortmund \hfill $\bullet$ \hfill Tel. 0231/3997729 \hfill}

%\pagenumbering{roman}
%\nopagenumbering
\newcounter{saveenumi}
% \renewcommand{\labelenumi}{\alph{enumi})}
% \renewcommand{\labelenumii}{\arabic{enumii}.}


\begin{document}
\lhead{ }
\label{ErsteSeite}
\centerline{ \Large \textbf{Curriculum vitae}}
\vspace{-.5cm}
%  \subsection*{General information}
%  \vspace{-.5cm}
\begin{tabbing}
\hspace{2.4cm} \= \\
	\>\textit{Stephan Sigg, Dipl.-Inform., Dr. rer. nat.}\\[.1cm]
% Date of birth \> January, 25th 1978 in Hanover, Germany\\[.1cm]
% Civil status \> Married, two daughters (2 and 5 years)\\[.1cm]
Residence	\> \Strasse, \Plz~\Ort, \Country\\[.1cm]
Phone \> \Telefon\\[.1cm]
Email \> \Email\\[.1cm]
Web \>http://www.stephansigg.de/stephan
\end{tabbing}
% \vspace{.5cm}



Since October 2013, Stephan Sigg is with the Computer Networks Group of Georg-August-University Goettingen. Before, he was a researcher at TU-Braunschweig and an academic guest in the Wearable Computing Lab at ETH Zurich and in the Nodes laboratory at University of Helsinki. From December 2010 to March 2013, Stephan was with the National Institute of Informatics (NII), in the Information systems architecture research division. He was a visiting Professor for Distributed and Ubiquitous Systems at the TU Braunschweig in the winter term 2010 and a PostDoc researcher at the chair for Pervasive Computing Systems (TecO) of the Karlsruhe Institute of Technology (KIT) in 2010 and a PostDoc researcher at the chair for Distributed and Ubiquitous Systems at the TU Braunschweig from 2008 to 2010. He obtained his PhD (Dr. rer. nat.; 2008) from University of Kassel where he was with the chair for Communication Technology (ComTec) from 2005 to 2007.

His research interests include the design, analysis and optimisation of algorithms for distributed and ubiquitous systems. 
Recently he has considered problems related to context prediction, collaborative transmission in Wireless sensor networks, context-based secure key generation, device-free passive activity recognition and computation of functions in wireless networks at the time of transmission. 

% \subsection*{Short CV}
% \vspace{-.5cm}
\begin{tabbing}
\hspace{3.2cm} \= \hspace{2.5cm} \=\\
10.2013 --	\>\begin{minipage}[t]{13.4cm}	Researcher at the Georg-August-Universität Göttingen, Germany, in the Computer Networks group.\end{minipage}\\[0.2cm]
04.2014 -- 08.2014 \>\begin{minipage}[t]{13.4cm}	Research visit at University of Helsinki, Finland, in the Nodes laboratory.\end{minipage}\\[0.2cm]
08.2013 -- 09.2013 \>\begin{minipage}[t]{13.4cm}	Academic Guest at ETH Zurich, Switzerland in the Wearable Computing Lab.\end{minipage}\\[0.2cm]
06.2013 -- 09.2013 \>\begin{minipage}[t]{13.4cm}	Researcher at TU Braunschweig.\end{minipage}\\[0.2cm]
10.2012 -- 03.2013 \>\begin{minipage}[t]{13.4cm}	Part-time lecturer at Waseda University, Tokyo, Japan.\end{minipage}\\[0.2cm]
12.2010 -- 03.2013 \>\begin{minipage}[t]{13.4cm}	Researcher at the National Institute of Informatics (NII) , Tokyo, Japan in the Information Systems Architecture Research Division.\end{minipage}\\[0.2cm]
10.2010 -- 03.2011 \>\begin{minipage}[t]{13.4cm}	Visiting professorship at the distributed and ubiquitous systems group of the TU Braunschweig.\end{minipage}\\[0.2cm]
04.2010 -- 09.2010 \>\begin{minipage}[t]{13.4cm}	Research staff member (Postdoc) at the Karlsruhe Institute of Technology (KIT) at the chair for Pervasive Computing Systems (TecO).\end{minipage}\\[0.2cm]
01.2008 -- 03.2010 \>\begin{minipage}[t]{13.4cm}	Research staff member (Postdoc) at TU Braunschweig in the distributed and ubiquitous systems group.\end{minipage}\\[0.2cm]
01.2005 -- 12.2007 \>\begin{minipage}[t]{13.4cm}	Research staff member at the University of Kassel at the chair for communication technology (ComTec).\end{minipage}
\end{tabbing}

\centerline{ \Large \textbf{Publications}}
 \vspace{1cm}
%  \subsection*{General information}
%  \vspace{-.5cm}
% \begin{tabbing}
% \hspace{2.4cm} \= \\
% 	\>\textit{Stephan Sigg, Dipl.-Inform., Dr. rer. nat.}\\[.1cm]
% % Date of birth \> January, 25th 1978 in Hanover, Germany\\[.1cm]
% % Civil status \> Married, two daughters (2 and 5 years)\\[.1cm]
% Residence	\> \Strasse, \Plz~\Ort, \Country\\[.1cm]
% Phone \> \Telefon\\[.1cm]
% Email \> \Email\\[.1cm]
% Web \>http://www.stephansigg.de/stephan
% \end{tabbing}
% % \vspace{.5cm}
% \section*{Publications}
Most significant publications of the last three years are highlighted.
\subsection*{Journals}
\sethlcolor{lightgray}
\begin{enumerate}
\item \hl{Stephan Sigg: A fast binary feedback-based distributed adaptive carrier synchronisation for transmission among clusters of disconnected IoT nodes in smart spaces, Elsevier Journal on Ad Hoc Networks, vol. 16, May 2014, pp. 120-130 (doi:10.1016/j.adhoc.2013.12.006)}
\item \hl{Shuyu Shi, Stephan Sigg, Wei Zhao, and Yusheng Ji: Monitoring of Attention from Ambient FM-radio Signals, IEEE Pervasive Computing, Los Alamitos, CA, USA, IEEE Computer Society, Jan-Mar 2014, vol. 13, no. 1, pp. 30-36, 2014 (doi:10.1109/MPRV.2014.13)}
\item Stephan Sigg, Shuyu Shi and Yusheng Ji: Teach your WiFi-Device: Recognise Simultaneous Activities and Gestures from Time-Domain RF-Features, International Journal on Ambient Intelligence and Computing (IGI), vol. 6, no. 1, January 2014 (doi:10.4018/ijaci.2014010102)
\item \hl{Stephan Sigg, Markus Scholz, Shuyu Shi, Yusheng Ji, Michael Beigl, "RF-Sensing of Activities From Non-Cooperative Subjects in Device-Free Recognition Systems Using Ambient and Local Signals," IEEE Transactions on Mobile Computing, Feb. 2014, vol. 13, no. 4 (doi:10.1109/TMC.2013.28)}
\item \hl{Dominik Schuermann, Stephan Sigg, "Secure Communication Based on Ambient Audio," IEEE Transactions on Mobile Computing, vol. 12, no. 2, pp.358-370, Feb., 2013 (doi: 10.1109/TMC.2011.271)}
\item Markus Scholz, Dawud Gordon, Leonardo Ramirez, Stephan Sigg, Tobias Dyrks, Michael Beigl: A Concept for Support of Firefighter Frontline Communication, in Future Internet, vol.5, no. 2, pp. 113-127, 2013 (doi: 10.3390/fi5020113)
\item \hl{Stephan Sigg, Dawud Gordon, Georg von Zengen, Michael Beigl, Sandra Haseloff, Klaus David, "Investigation of Context Prediction Accuracy for Different Context Abstraction Levels," IEEE Transactions on Mobile computing, vol.11, no.6, pp.1047-1059, June 2012 (doi: 10.1109/TMC.2011.170)}
\item Predrag Jakimovski, Hedda R. Schmidtke, Stephan Sigg, Leonardo Weiss Ferreira Chaves, Michael Beigl, "Collective Communication for Dense Sensing Environments," Journal of Ambient Intelligence and Smart Environments, IOS Press, vol. 4, no. 2, pp.123-134, 2012 (doi: 10.3233/AIS-2012-0139)
\item \hl{Stephan Sigg, Rayan Merched El Masri, Michael Beigl, "Feedback-Based Closed-Loop Carrier Synchronization: A Sharp Asymptotic Bound, an Asymptotically Optimal Approach, Simulations, and Experiments," IEEE Transactions on Mobile Computing, vol.10, no.11, pp.1605-1617, Nov. 2011 (doi: 10.1109/TMC.2011.21)}
\item Stephan Sigg, Sandra Haseloff, Klaus David, "An Alignment Approach for Context Prediction Tasks in UbiComp Environments," IEEE Pervasive Computing, vol.9, no.4, pp.90-97, October-December 2010 (doi: 10.1109/MPRV.2010.23)
\setcounter{saveenumi}{\value{enumi}}
\end{enumerate}
\subsection*{Conferences and workshops}
\begin{enumerate}
\setcounter{enumi}{\value{saveenumi}}
\item Stephan Sigg, Xiaoming Fu: Social Opportunistic Sensing and Social Centric Networking - Enabling technology for Smart Cities, ACM International Workshop on Wireless and Mobile Technologies for Smart Cities (WiMobCity 2014), in conjunction with MobiHoc 2014, Philadelphia, PA, USA, ACM, August 2014.
\item \hl{Stephan Sigg, Ulf Blanke and Gerhard Troester: The Telepathic Phone: Frictionless Activity Recognition from WiFi-RSSI, IEEE International Conference on Pervasive Computing and Communications (PerCom), Budapest, Hungary, March 24-28, 2014}
\setcounter{saveenumi}{\value{enumi}}
\end{enumerate}
% \pagebreak
\lhead{\tiny \Name, \Strasse, \Plz\ \Ort, \Country. Phone: \Telefon, Email: \Email}
\begin{enumerate}
\setcounter{enumi}{\value{saveenumi}}
% \item Shuyu Shi, Stephan Sigg, Wei Zhao, and Yusheng Ji: Monitoring of Attention from Ambient FM-radio Signals, IEEE Pervasive Computing, Los Alamitos, CA, USA, IEEE Computer Society, Jan-Mar 2014, vol. 13, no. 1, pp. 30-36, 2014
% \item Stephan Sigg, Markus Scholz, Shuyu Si, Yusheng Ji and Michael Beigl: RF-sensing of activities from non-cooperative subjects in device-free recognition systems using ambient and local signals, in IEEE Transactions on Mobile Computing (TMC), Feb. 2013, vol. 13, no. 4 
\item Jiyin He, Kai Kunze, Christoph Lofi, Sanjay K. Madria and Stephan Sigg: Towards Mobile Sensor-Aware Crowdsourcing: Architecture, Opportunities and Challenges, UnCrowd 2014: DASFAA Workshop on Uncertain and Crowdsourced Data, Bali, Indonesia, April 21, 2014
\item Stephan Sigg, Mario Hock, Markus Scholz, Gerhard Troester, Lars Wolf, Yusheng Ji and Michael Beigl: Passive, device-free recognition on your mobile phone: tools, features and a case study, 10th International Conference on Mobile and Ubiquitous Systems: Computing, Networking and Services (Mobiquitous 2013), Tokyo, Japan, 2013
\item Stephan Sigg, Shuyu Shi, Felix Buesching, Yusheng Ji and Lars Wolf: Leveraging RF-channel fluctuation for activity recognition, 11th International Conference on Advances in Mobile Computing and Multimedia (MoMM2013), Vienna, Austria, 2013
\item Stephan Sigg, Shuyu Shi, and Yusheng Ji: RF-based device-free recognition of simultaneously conducted activities, adjunct proceedings of the 2013 ACM International Joint Conference on Pervasive and Ubiquitous Computing (UbiComp 2013), Zurich, Switzerland, 2013
\item Shuyu Shi, Stephan Sigg and Yusheng Ji: Joint Localisation and Activity Recognition from Ambient FM Broadcast Signals, adjunct proceedings of the 2013 ACM International Joint Conference on Pervasive and Ubiquitous Computing (UbiComp 2013), Zurich, Switzerland, 2013
\item Shuyu Shi, Stephan Sigg and Yusheng Ji: ActiviTune: A Multi-stage system for activity recognition of passive entities from ambient FM-radio signals, in the 8th International Conference on Wireless Algorithms, Systems, and Applications (WASA 2013)
\item Stephan Sigg, Predrag Jakimovski, Yusheng Ji and Michael Beigl: Utilising an algebra of random functions to realise function calculation via a physical channel, accepted for presentation at the 2013 IEEE 14th Workshop on Signal Processing Advances in Wireless Communications (SPAWC)
\item Shuyu Shi, Stephan Sigg, and Yusheng Ji, "Passive detection of situations from ambient FM-radio signals" In Proceedings of the 2012 ACM Conference on Ubiquitous Computing (UbiComp '12). Pittsburgh, PA, USA, 1049-1053, 2012 (doi: 10.1145/2370216.2370440) 
\item \hl{Stephan Sigg, Predrag Jakimovski, Michael Beigl, "Calculation of functions on the RF-channel for IoT," 2012 3rd International Conference on the Internet of Things (IOT), pp.107-113, 24-26 Oct. 2012 (doi: 10.1109/IOT.2012.6402311)}
\item Shuyu Shi; Stephan Sigg, Yusheng Ji, "Activity recognition from radio frequency data: Multi-stage recognition and features," 2012 IEEE Vehicular Technology Conference (VTC Fall), pp.1-6, 3-6 Sept. 2012 (doi: 10.1109/VTCFall.2012.6399382)
\item Stephan Sigg, Lei Zhong, Yusheng Ji, "Activity recognition with implicit context classification," 2012 IEEE Vehicular Technology Conference (VTC Fall), pp.1-6, 3-6 Sept. 2012 (doi: 10.1109/VTCFall.2012.6399379)
\item Stephan Sigg, Ngu Nguyen, An Huynh and Yusheng Ji: AdhocPairing: Spontaneous audio based secure device pairing for Android mobile devices, in Proceedings of the 4th International Workshop on Security and Privacy in Spontaneous Interaction and Mobile Phone Use, in conjunction with Pervasive 2012, 2012
\item \hl{Ngu Nguyen; Stephan Sigg, An Huynh, Yusheng Ji, "Pattern-Based Alignment of Audio Data for Ad Hoc Secure Device Pairing," 2012 16th International Symposium on Wearable Computers (ISWC), pp.88-91, 18-22 June 2012 (doi: 10.1109/ISWC.2012.14)}
\setcounter{saveenumi}{\value{enumi}}
\end{enumerate}
\begin{enumerate}
\setcounter{enumi}{\value{saveenumi}}
\item Ngu Nguyen; Stephan Sigg, An Huynh, Yusheng Ji, "Using ambient audio in secure mobile phone communication," 2012 IEEE International Conference on Pervasive Computing and Communications Workshops (PERCOM Workshops), pp.431-434, 19-23 March 2012 (doi: 10.1109/PerComW.2012.6197527)
\item Markus Scholz, Stephan Sigg, Hedda Schmidtke and Michael Beigl: Challenges for device-free radio-based activity recognition, in Proceedings of the 3rd Workshop on Context Systems Design Evaluation and Optimisation (CoSDEO), 2011
\item Markus Scholz, Stephan Sigg, Dimana Shihskova, Georg von Zengen, Gerrit Bagshik, Toni Guenther, Michael Beigl and Yusheng Ji: SenseWaves: Radiowaves for context recognition, in Video Proceedings of the 9th International Conference on Pervasive Computing (Pervasive 2011), 2011
\item Stephan Sigg, Dominik Schuermann and Jusheng Ji: PINtext: A framework for secure communication based on context, in Proceedings of the 8th International ICST Conference on Mobile and Ubiquitous Systems: Computing, Networking, and Services (MobiQuitous 2011), Lecture Notes of the Institute for Computer Sciences, Social Informatics and Telecommunications Engineering, vol. 104, pp.314-325, 2011 (doi:10.1007/978-3-642-30973-1\_31)
\item Stephan Sigg, Predrag Jakimovski, Florian Becker, Hedda Schmidtke, Martin Alexander Neumann, Yusheng Ji and Michael Beigl: Neuron-inspired collaborative transmission in wireless sensor networks, in Proceedings of the 8th International ICST Conference on Mobile and Ubiquitous Systems: Computing, Networking, and Services (MobiQuitous 2011), Lecture Notes of the Institute for Computer Sciences, Social Informatics and Telecommunications Engineering, vol. 104, pp.273-284, 2011 (doi: 10.1007/978-3-642-30973-1\_28)
\item Stephan Sigg, "Context-based security: state of the art, open research topics and a case study". In Proceedings of the 5th ACM International Workshop on Context-Awareness for Self-Managing Systems (CASEMANS '11). 	ACM, New York, NY, USA, 17-23. (doi: 10.1145/2036146.2036150) 
% Stephan Sigg, Matthias Budde, Yusheng Ji and Michael Beigl: Entropy of audio fingerprints for unobtrusive device authentication, in Poster Proceedings of the 7th International and Interdisciplinary Conference on Modeling and Using Context (Context2011), 2011 (Online)
\item Predrag Jakimovski, Florian Becker, Stephan Sigg, Hedda Rahel Schmidtke, Michael Beigl, "Collective Communication for Dense Sensing Environments," 2011 7th International Conference on Intelligent Environments (IE), pp.157-164, 25-28 July 2011 (doi: 10.1109/IE.2011.42) (**Best paper**)
\item Behnam Banitalebi, Dawud Gordon, Stephan Sigg, Takashi Miyaki, Michael Beigl, "Collaborative Channel Equalization: Analysis and Performance Evaluation of Distributed Aggregation Algorithms in WSNs," 2011 8th IEEE International Conference on Mobile Ad-hoc and Sensor Systems (MASS), pp.450-459, 17-22 Oct. 2011 (doi: 10.1109/MASS.2011.51)
\item Markus Reschke, Sebastian Schwarzl, Johannes Starosta, Stephan Sigg, "Situation Awareness Based on Channel Measurements," 2011 73rd IEEE Vehicular Technology Conference (VTC Spring), pp.1-5, 15-18 May 2011 (doi: 10.1109/VETECS.2011.5956453)
\item Stephan Sigg and Michael Beigl: An adaptive protocol for distributed beamforming, in 17. Fachtagung 'Kommunikation in Verteilten Systemen 2011' (KiVS11), vol. 17 of OASICS, pp.38-48, Schloss Dagstuhl - Leibnitz-Zentrum fuer Informatik, Germany, March 2011
\item Dawud Gordon, Stephan Sigg, and Michael Beigl, "Using prediction to conserve energy in recognition on mobile devices," 2011 IEEE International Conference on Pervasive Computing and Communications Workshops (PERCOM Workshops), pp.364-367, 21-25 March 2011 (doi: 10.1109/PERCOMW.2011.5766907)
\item Johannes Starosta, Markus Reschke, Sebastian Schwarzl, Stephan Sigg and Michael Beigl: Context awareness through the RF-channel, 24th International Conference on Architecture of Computing Systems (ARCS), Lecture Notes in Computer Science (LNCS), Springer, vol. 6566, Como, Italy, February 2011
\item Stephan Sigg, Michael Beigl and Behnam Banitalebi, "Efficient adaptive communication from resource restricted transmitters", In Organic Computing - A Paradigm Shift for Complex Systems, Christian Mueller-Schloer and Hartmut Schmeck and and Theo Ungerer, Springer, 2011
\item Behnam Banitalebi, Stephan Sigg and Michael Beigl, "Performance analysis of receive collaboration in TDMA-based Wireless Sensor Networks", in Proceedings of the fourth International Conference on Mobile Ubiquitous Computing, Systems, Services and Technologies (Ubicomm), October 2010
\item Stephan Sigg and Michael Beigl, "Expectation aware in-network context processing". In Proceedings of the 4th ACM International Workshop on Context-Awareness for Self-Managing Systems (CASEMANS '10). ACM, New York, NY, USA, 2010, (doi: 10.1145/1858367.1858376)
\item Behnam Banitalebi, Stephan Sigg, Michael Beigl, "On the feasibility of receive collaboration in wireless sensor networks," 2010 IEEE 21st International Symposium on Personal Indoor and Mobile Radio Communications (PIMRC), pp.1608-1613, 26-30 Sept. 2010 (doi: 10.1109/PIMRC.2010.5671943)
\item Rayan Merched El Masri, Stephan Sigg, Michael Beigl, "An asymptotically optimal approach to the distributed adaptive transmit beamforming in wireless sensor networks," 2010 European Wireless Conference (EW), pp.511-518, 12-15 April 2010 (doi: 10.1109/EW.2010.5483485)
\item Niklas Klein, Stephan Sigg, Klaus David and Michael Beigl, "DAG Based Context Reasoning: Optimised DAG Creation," 2010 23rd International Conference on Architecture of Computing Systems (ARCS), pp.1-6, 22-23 Feb. 2010
\item Stephan Sigg, Rayan Merched El Masri, Julian Ristau and Michael Beigl, "Limitations, performance and instrumentation of closed-loop feedback based distributed adaptive transmit beamforming in WSNs," 5th International Conference on Intelligent Sensors, Sensor Networks and Information Processing (ISSNIP), pp.451-456, 7-10 Dec. 2009 (doi: 10.1109/ISSNIP.2009.5416750)
\item Stephan Sigg, Michael Beigl, "Algorithmic approaches to distributed adaptive transmit beamforming," 5th International Conference on Intelligent Sensors, Sensor Networks and Information Processing (ISSNIP), pp.433-438, 7-10 Dec. 2009 (doi: 10.1109/ISSNIP.2009.5416780)
\item Stephan Sigg, Michael Beigl, "Algorithms for closed-loop feedback based distributed adaptive beamforming in wireless sensor networks," 5th International Conference on Intelligent Sensors, Sensor Networks and Information Processing (ISSNIP), pp.25-30, 7-10 Dec. 2009 (doi: 10.1109/ISSNIP.2009.5416847)
\item Stephan Sigg, Michael Beigl, "Randomised Collaborative Transmission of Smart Objects", in 2nd International Workshop on Design and Integration Principles for Smart Objects (DIPSO2008) in conjunction with Ubicomp 2008, September 2008
% \item Stephan Sigg, Daniel Röhr, Monty Beuster and Michael Beigl: Audio Fingerprinting in UbiComp Environments - Performance Measurements and Applications, in Proceedings of the fifth International Conference on Networked Sensing Systems (INSS'08), June 2008
% \item Stephan Sigg, Monty Beuster, Daniel Röhr and Michael Beigl: Search Space Size and Context Prediction, in Proceedings of the fifth International Conference on Networked Sensing Systems (INSS'08), June 2008
\item Stephan Sigg and Michael Beigl, "Collaborative Transmission in Wireless Sensor Networks by a ($1+1$)-EA," Eighth International Workshop on Applications and Services in Wireless Networks, ASWN '08. pp.37-46, 9-10 Oct. 2008 (doi: 10.1109/ASWN.2008.12)
\item Stephan Sigg, Sandra Haseloff, Klaus David, "A study on context prediction and adaptivity," 2nd International Conference on Digital Information Management, ICDIM '07, vol.2, pp.717-722, Lyon, France, 28-31 Oct. 2007 (doi: 10.1109/ICDIM.2007.4444309)
\item Stephan Sigg, Sandra Haseloff, Klaus David, "Prediction of Context Time Series", Proceedings of the 5th Workshop on Applications of Wireless Communications (WAWC'07), August 16, Lappeenranta, Finland, 2007
\item Stephan Sigg, Sian Lun Lau, Sandra Haseloff, and Klaus David, "Approaching a definition of context prediction", Proceedings of the 3rd Workshop on Context Awareness for Proactive Systems CAPS07, Guildford, June 2007
\item Stephan Sigg, Sandra Haseloff, Klaus David, "Minimising the Context Prediction Error,"  65th IEEE Vehicular Technology Conference, VTC2007-Spring, pp.272-276, 22-25 April 2007\linebreak (doi: 10.1109/VETECS.2007.68)
\item Stephan Sigg, Sandra Haseloff, Klaus David, "A Novel Approach to Context Prediction in UBICOMP Environments," 2006 IEEE 17th International Symposium on Personal, Indoor and Mobile Radio Communications, pp.1-5, Helsinki, Finland, 11-14 Sept. 2006 (doi: 10.1109/PIMRC.2006.254051)
\item Stephan Sigg, Sandra Haseloff, Klaus David, "The Impact of the Context Interpretation Error on the Context Prediction Accuracy," 3rd Annual International Conference on Mobile and Ubiquitous Systems - Workshops, pp.1,4, San Jose, CA, 17-21 July 2006 (doi: 10.1109/MOBIQW.2006.361718)
\item Tino Löffler, Stephan Sigg, Sandra Haseloff, Klaus David: The Quick Step to Foxtrot. In: K. David, O. Drögehorn, S. Haseloff (eds.): Proceedings of the Second Workshop on Context Awareness for Proactive Systems (CAPS 2006), June 12-13, Kassel, Germany. Kassel university press, 2006
\item Stephan Sigg, Klaus David, "Optimum Resource Allocation in HSDPA," 12th European Wireless Conference 2006 - Enabling Technologies for Wireless Multimedia Communications (European Wireless), pp.1-7, Athen, Greece, 2-5 April 2006
\setcounter{saveenumi}{\value{enumi}}
\end{enumerate}

\subsection*{Theses and books}
\begin{enumerate}
\setcounter{enumi}{\value{saveenumi}}
\item Stephan Sigg: Optimisation of a three stage cooling process, Master's Thesis, FernUniversität in Hagen, February 2010
\item Stephan Sigg: Ein Vergleich von Varianten endlicher Quantenautomaten - Eine algorithmenorientierte Analyse, VDM, March 2008 (in German)
\item Stephan Sigg: Development of a novel context prediction algorithm and analysis of context prediction schemes. PhD thesis, University of Kassel, Chair for Communication Technology, February 2008
\item Stephan Sigg: Ein Vergleich verschiedener Varianten endlicher Quanten-Automaten. Diploma thesis, University of Dortmund, August 2004 (in German)
\end{enumerate}


\label{MaxSeitenzahl}
\end{document}
