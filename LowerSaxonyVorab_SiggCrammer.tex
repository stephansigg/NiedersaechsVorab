\documentclass[12pt]{article}
\usepackage{setspace}

\usepackage{ucs}
\usepackage[utf8x]{inputenc}
%\usepackage{ngerman}
\usepackage{a4wide}
\usepackage{amsmath}
\usepackage{amsthm}
\usepackage{color}
\definecolor{light}{gray}{0.85}
\definecolor{heavy}{gray}{0.30}
\usepackage{amsfonts}
\usepackage{dsfont}
\usepackage{amssymb}
\usepackage{setspace}% used for table notes
\usepackage{makeidx}
\usepackage[pdftex]{graphicx}
\usepackage{multibib}
% \usepackage{subfigure}
\usepackage{subfig}
\usepackage{eurosym}
\newcites{all,own,related}%
         {\normalsize{References (other authors)},%
          \normalsize{References (applicant)},
          \normalsize{ }\vspace{-.8cm}}
          
\usepackage{color}
          
\newcommand{\todoBase}{\hfill \notiz{TODO} $\square$}
\newcommand{\projektname}{\texttt{Sense box}}

\newcommand{\notiz}[1]{\marginpar{\tiny \framebox{\begin{minipage}{2.1cm} #1 \end{minipage}}}}
\newcommand{\niklas}[1]{\marginpar{\footnotesize\notiz{Niklas:}\\\tiny \framebox{\begin{minipage}{2.1cm} #1 \end{minipage}}}}
\newcommand{\dawud}[1]{\marginpar{\footnotesize\notiz{Dawud:}\\\tiny \framebox{\begin{minipage}{2.1cm} #1 \end{minipage}}}}
\newcommand{\stephan}[1]{\marginpar{\footnotesize\notiz{Stephan:}\\\tiny \framebox{\begin{minipage}{2.1cm} #1 \end{minipage}}}}
\newcommand{\todo}{\marginpar{\todoBase}}

\newcommand{\kobyc}[1]{\begin{center}\fbox{\parbox{3in}{{\textcolor{green}{K: #1}}}}\end{center}}
\newcommand{\stephans}[1]{\begin{center}\fbox{\parbox{3in}{{\textcolor{magenta}{S: #1}}}}\end{center}}

\newcommand{\nolineskips}{
	\setlength{\parskip}{0pt}
	\setlength{\parsep}{0pt}
	\setlength{\topsep}{0pt}
	\setlength{\partopsep}{0pt}
	\setlength{\itemsep}{0pt}}


\author{\begin{minipage}[t]{7.1cm}\centering \small \VornameAntragstellerA\ \NachnameAntragstellerA\\ \small Lehrstuhl für Kommunikationstechnik\\ \small ComTec\\ \small Universität Kassel\end{minipage}
\begin{minipage}[t]{7.1cm}\centering \small \VornameAntragstellerB\ \NachnameAntragstellerB\\ \small Gruppe Verteilte und Ubiquitäre Systeme \\ \small Institut für Betriebssysteme und Rechnerverbund\\ \small TU Braunschweig\end{minipage}}

\title{\projektname:\\\notiz{Social and sentiment sensing and assisting using on-body sensors}}

\date{\small \today}
% \usepackage[T1]{fontenc}
% \newcommand{\changefont}[3]{
% \fontfamily{#1} \fontseries{#2} \fontshape{#3} \selectfont}
%\changefont{cmss}{m}{n}
\fontfamily{cmss}
\fontseries{m}
\fontshape{n}
\selectfont
%%serifenlose Schrift:
\renewcommand{\familydefault}{\sfdefault}
\usepackage{helvet}

%% Use header as in DFG template
 \usepackage{fancyhdr}

%% Redefine plain page style
\fancypagestyle{plain}{
 \fancyhf{}
 \renewcommand{\headrulewidth}{0pt}
 \fancyfoot[LE,RO]{\thepage}
 }
 \pagestyle{plain}
% \lhead{\tiny \Name, \Strasse, \Plz\ \Ort, Germany. Phone: \Telefon, Email: \Email}
\rhead{}
% \rhead{\small page \arabic{page} of \pageref{maxSeitenzahl}}
\rfoot{}
%% Linebreak after paragraph
\makeatletter
\renewcommand\paragraph{\@startsection{paragraph}{4}{\z@}%
  {-3.25ex\@plus -1ex \@minus -.2ex}%
  {1.5ex \@plus .2ex}%
  {\normalfont\normalsize\bfseries}}
\makeatother
\begin{document}
\onehalfspacing %1.5 spacing
%% enable numbering for paragraphs also:
\setcounter{secnumdepth}{5}
% \maketitle
\pagebreak
\begin{center}
\section*{Social and sentiment sensing and assisting using on-body sensors}
\begin{tabular}{cc}
  Stephan Sigg, & Koby Crammer\\
  University of Goettingen, Germany & Department of Electrical Engineering, Technion \\
 stephan.sigg@cs.uni-goettingen.de & koby@ee.technion.ac.il
\end{tabular}

\end{center}
\begin{description}
	\item[Keywords] \textit{Sentiment sensing, Mobile sensing, Machine learning, Data analysis}
\end{description}

\subsection*{Topic, objectives and justification of the project}
Emotional states are closely linked to Physical health~\cite{SentimentSensing_Salovey_2000}, such as anger, which increases the danger of heart disease~\cite{SentimentSensing_Smith_2004} or mood which impacts health-relevant behavior~\cite{SentimentSensing_Todaro_2003}.
In addition, mental health is linked to emotion regulation~\cite{SentimentSensing_Gross_1995}, whereas stress, especially when pertaining, has significant impact on brain, cognition and work performance~\cite{SentimentSensing_Lupien_2009}.
On the other hand, positive affect benefits the immune system~\cite{SentimentSensing_Smith_2004} while happiness and satisfaction are good indicators for work performance, health, and also success in private life~\cite{SentimentSensing_Lyubomirsky_2005}.

Sentiment and emotion can be detected from Physiological signals with sensors such as Electromyograms (EMG, muscle activity), Electrodermal Activity (EDA, electrical conductivity of the skin surface), Electrocardiogram (EKG or ECG, heart activity), Electrooculogram (EOG, eye movement) or Electroencephalography (EEG, brain activity)~\cite{SentimentSensing_Calvo_2010}.
These kind of sensors, however, are accessible only to a small share of the population and have drawbacks such as high cost, weight, non-portability, and sometimes involve invasive implantation.
% Furthermore, for a human utilizing these tools, they constitute additional devices to carry around.

Recent work shows that it is also possible to detect sentiment from gesture and pose~\cite{Emotion_DeMeijer_1989}. 
With the verge of the Internet of Things and increasing amount of mobile and body worn devices and sensors in everyday life, a multitude of sensors is available nowadays which has the potential to capture body movement and pose. 

The focus and contribution of this project is to (1) recognise emotion from medical sensors, (2) exploit the potential of cheaper smartphone and on-body IoT sensors for sentiment sensing by identifying correlations in sensor data traces with medical sensors, (3) exploit environmental signals such as RSSI or RF-signal strength for sentiment sensing, (4) predict the evolution of sentiment from the time series of past observations, (5) develop memory, time, and computation efficient methods to perform such analysis on a smartphone and, (6) develop and experiment with sentiment feedback methods.

The outcome of the project will be a sentiment sensing and feedback framework comprising medical, body, smartphone and environmental sensors together with a prototype implementation

\subsection*{Current state-of-the-art in an international perspective}
Emotion research or affective science investigates affect expression and detection~\cite{SentimentSensing_Calvo_2010}.
In this context, the recognition of emotion has been considered from facial expressions~\cite{SentimentSensing_Ekman_2003}, voice~\cite{SentimentSensing_Juslin_2005} as well as body language and posture~\cite{SentimentSensing_Mota_2003,SentimentSesing_Mello_2009}.
Especially the latter is of greater interest for this project and is discussed below. 
There is also some body of work on utilizing text mining for sentiment sensing as briefly discussed in the next section. 

\subsubsection*{Sensing of sentiment from textual corpus}
Sentiment mining is an active research topic in recent years~\cite{SentimentSensing_Kramer_2014, SentimentSensing_Cambria_2013}. 
Also referred to as opinion mining, it considers indirect sources and requires actual active involvement of subjects~\cite{SentimentSensing_Pang_2008, SentimentSensing_Zhai_2011}. 
In particular, there is an active body of work in the direction of mining textual input (via sentiment-expressing words or phrases) in online social networks with the help of natural language processing and computational linguistics~\cite{SentimentSensing_Liu_2012}.
We will, however, not consider the mining of text in the scope of this project but rather focus on sentiment sensing from physical sensors. Nevertheless, if necessary, we may exploit textual input to better tune the parameters of our sensor-based sentiment analyzer. 

\subsubsection*{Sentiment sensing from body movement and pose}
Emotion can be inferred from body gesture and pose~\cite{Pervasive_Jaggarwal_2012} at least as accurately as from face~\cite{Emotion_Nguyen_2012,Emotion_Castellano_2008,Emotion_Meeren_2005}.
The role of human body in emotion expression has received support through evidences from psychology~\cite{Emotion_Walters_1986} and nonverbal communication \cite{Emotion_Dittmann_1978}. 
The importance of  bodily expression has also been confirmed for emotion detection \cite{Emotion_Wallbott_1998, Emotion_VanHeijnsbergen_2007,Emotion_Atkinson_2007}. 
%Detection of Fear: Emotion_VanHeijnsbergen_2007
Walter and Walk~\cite{Emotion_Walters_1986} revealed that emotion recognition from photos of postural expression, in which faces and hands were covered, was as accurate as recognition from facial expression alone.
% In another study, Meeren et al.~\cite{Emotion_Meeren_2005} indicated that observers were strongly influenced by body posture when judging a facial expression. 
Dynamic configurations of human body even hold greater amount of information as indicated by Bull in \cite{Emotion_Bull_1987}. 
He proved that body positions and motions could be recognised in terms of states including interest/boredom and agreement/disagreement. 
Some other studies went further by looking for the contribution of separated body parts to particular emotional states~\cite{Emotion_DeMeijer_1989,Emotion_Montepare_1999}.
Emotion can be recognised from non-trivial scenarios, such as simple daily-life actions \cite{Emotion_Crane_2007,Emotion_Bernhardt_2007} or recognition ability of infants \cite{Emotion_Lagerlof_2009}. 
% In~\cite{Emotion_Crane_2007}, Crane and Gross tested the ability of  recognizing 5 emotions from gait actions. 

% It will be interesting to see how well RF information can be exploited in order to identify body gesture and pose and to classify this for human emotion classes.

 Researchers~\cite{SentimentSensing_Mota_2003,SentimentSesing_Mello_2009}, were able to detect boredom, confusion, delight and frustration utilizing pressure mats. 
Such mats are not portable and thus are not applicable in our setting. 
Instead, we aim to detect sentiment from body- or environmental sensors. 

% \paragraph*{Sensing attention from environmental sensors}
Additionally, the sensing of attention has been investigated previously. 
Attention is an important measure in Computer-Human interaction. 
It determines for an interactive system the potential to impact the actions and decisions taken by an individual~\cite{AttentionMonitoring_Xu_2012}.
In the literature, we find various definitions that classify attention as well as its determining characteristics~\cite{AttentionMonitoring_Wu_2007, AttentionMonitoring_Wickens_1984}.
While the tracking of gaze is a commonly utilized measure of attention~\cite{AttentionMonitoring_Yonezawa_2007}, also other observable features may indicate attention. 
In general, aspects such as saliency, effort, expectancy and value are important indicators of attention~\cite{AttentionMonitoring_Wickens_2008, AttentionMonitoring_Wickens_1984,AttentionMonitoring_Xu_2012,AttentionMonitoring_Gollan_2011}.
This model was later extended to put a greater stress on the effort a person takes towards an object~\cite{AttentionMonitoring_Ferscha_2012}.
The authors also discuss various aspects of attention and identify as most distinguishing factors changes in walking speed, direction or orientation.

Shi et al~\cite{Pervasive_Shi_2014} investigated how these properties, in particular location of a person, walking direction and walking speed or changes therein can be utilized for the monitoring and detection of attention from fluctuation in received signal strength. 
%% TODO: Add work of Kai Kunze? -> Fatigue, eye movement,...

\subsubsection*{Sensing activity, motion and pose from inertial and environmental sensors}
Activity recognition comprises the challenge to recognize human activities from the input of sensor data.
Traditionally, acceleration has evolved as the standard modality for activity recognition both for high diffusion and convincing recognition rates~\cite{Pervasive_Ravi_2005,Pervasive_Cao_2012}.
General research challenges for activity recognition regard the accurate classification of noisy data captured under real world conditions~\cite{Pervasive_Bao_2004} or the automation of recognition systems~\cite{Pervasive_Ploetz_2012}.
The classification accuracy is highly dependent on the accurate sensor placement.
The integration of sensors in clothing as well as the recent remarkable progress in the robustness to rotation or displacement have improved this situation greatly~\cite{Pevasive_Chavarriaga_2011}.
% However, a subject is still required to cooperate and at least wear the sensors~\cite{Pervasive_Cohn_2012}.
% This requirement can not be assured generally in real-world applications. 
% In particular, even devices as private as mobile phones, which are frequently assumed to be constantly in the same context as its owner~\cite{Pervasive_Patridge_2008, Pervasive_Varshavsky_2008, Pervasive_Lane_2010}, can not serve as a sensor platform suitable for continuous monitoring of an individual's context.   
% Dey et al.~\cite{Pervasive_Dey_2011} showed that users have their mobile phone within arms reach only 54\% of the time.
% This confirms a similar investigation of Patel et al.~\cite{Pervasive_Patel_2006} which reported a share of 58\% for the same measure.

We will in addition consider environmental sources, in particular signal strength information from WiFi or 2-3G systems for their ubiquitous availability. 
In particular, good accuracy has been achieved recently with systems exploiting fluctuation in received radio signals. 
Pu and others~\cite{RFsensing_Pu_2013,RFsensing_Kim_2009} showed that simultaneous detection of gestures from multiple individuals is possible by utilizing multi-antenna nodes and micro Doppler fluctuations.
They utilize a USRP SDR multi antenna receiver and one or more single antenna transmitters distributed in the environment to distinguish between a set of 9 gestures with an average accuracy of 0.94. 
Their active device-free system exploits a MIMO receiver in order to recognise gestures from different persons present at the same time. 
By leveraging a preamble gesture pattern, the receiver estimates the MIMO channel that maximises the reflections of the desired user.

A main challenge was for them that the Doppler shift from human movement is several magnitudes smaller than the bandwidth of the signal employed.
The authors therefore proposed to transform the received signal into several narrowband pulses which are then analysed for possible Doppler fluctuation.
The group recently discussed application possibilities of their system~\cite{RFSensing_Kellog_2014}.

In a related system, Adib and Katabi~\cite{Pervasive_Adib_2013} employ MIMO interference nulling and combine samples taken over time to achieve a similar result while compensating for the missing spatial diversity in a single-antenna receiver system.
In their system, they leverage standard WiFi hardware at 2.4GHz.

Later, this work was extended to 3D motion tracking by utilizing three or more directional receive antennas in exactly defined relative orientation~\cite{RFSensing_Adib_2014}. 
In particular, the system is able to track the center of a human body with an error below 21cm in any direction and can also detect movement of body parts and directions of a pointing body part, such as a hand. 
This localisation is possible through time-of-flight estimation and triangulation.
Higher accuracy of this estimation is granted by utilizing frequency modulated carrier waves (sending a signal that changes linearly in frequency with time) over a bandwidth of 1.69GHz.
Impact of static objects could be mitigated by subtracting successive sample means whereas noise was filtered by its speed of changes in energy over frequency bands. 

\subsection*{Research approach, methods and hypotheses}
We plan to apply a machine learning approach in which prediction and feedback tools are built based on data. Specifically, we plan to install and employ a wide range of sensors, some on-body, some are already installed on everyday smartphones, and some are located in an environment. 
We will collect data from these sensors and analyse it, where the goal is to find a small, cheap and efficient subset of sensors which provides a robust and stable signal for prediction. Next, we will build a prediction system of sentiment based on this signal. We will use the collected signals and true sentiment to generate a predictive model of sentiment from the signal.
We will test the model on signals coming from participants of our study. We will also develop a feedback algorithm that will provide user information about current sentimental state, and possibly future predicted one. Hopefully, the user will benefit from such feedback. Additional components we plan to pursue is the development of efficient prediction and feedback methods, that could be executed on a smartphone.

We hypothesize that we will be able to collect sensor information of users, and their true or approximated sentimental state. We also assume that it is possible to predict sentiment from signals of sensors. Both assumptions are reasonable as similar tasks were  already performed by both PIs in other contexts. 

\subsection*{Detailed work programme including time schedule}
The expected time schedule is depicted in table~\ref{tableGesamtueberblick}. 
A detailed description of the distribution of work packages to personnel is given in the following.
The figures represent three person months (equal to one person quarter).
The notation '40' represents an allocation of 40 percent of one person quarter. 

\begin{table}
\centering
\begin{footnotesize}
\begin{tabular}{|l||c|c|c|c|c|c|c|c|c|c|c|c||c|}
 \hline
% Workpackages&&&&&&&&&&&&&\\
	& Q1	&Q2	&Q3	&Q4	&Q5	&Q6	&Q7	&Q8	&Q9	&Q10	&Q11	&Q12	&\\\hline\hline
Recent advances& &	&	&	&	&	&	&	&	&	&	& 	&\\
WP-1	&25	&25	&20	&20	&20	&20	&20	&15	&10	&10	&10	&5 	& 200\\\hline
Sensing system& &	&	&	&	&	&	&	&	&	&	& 	&\\
WP-2.1	&100	&	&	&	&	&	&	&	&	&	&	& 	& 100\\
WP-2.2	&50	&50	&	&	&	&	&	&	&	&	&	& 	& 100\\
WP-2.3	&25	&75	&	&	&	&	&	&	&	&	&	& 	& 100\\
WP-2.4	&	&50	&50	&	&	&	&	&	&	&	&	& 	& 100\\
WP-2.5	&	&	&	&30	&30	&40	&	&	&	&	&	& 	& 100\\\hline
Experiments& &	&	&	&	&	&	&	&	&	&	&	&\\
WP-3.1	&	&	&130	&70	&	&	&	&	&	&	&	& 	& 200\\
WP-3.2	&	&	&	&	&	&	&50	&30	&30	&45	&30	&15 	& 200\\
WP-3.3	&	&	&	&	&	&	&	&	&60	&	&	&40 	& 100\\\hline
Analysis& &	&	&	&	&	&	&	&	&	&	& 	&\\
WP-4.1	&	&	&	&80	&20	&	&	&	&	&	&	& 	& 100\\
WP-4.2	&	&	&	&	&130	&70	&	&	&	&	&	& 	& 200\\
WP-4.3	&	&	&	&	&	&70	&130	&	&	&	&	& 	& 200\\
WP-4.4	&	&	&	&	&	&	&	&100	&55	&45	&	& 	& 200\\
WP-4.5	&	&	&	&	&	&	&	&	&	&50	&100	&50 	& 200\\\hline
Prediction& &	&	&	&	&	&	&	&	&	&	& &\\
%WP-5	&	&	&	&	&	&45	&55	&100	 & 200\\\hline
WP-5.1	&	&	&	&	&	&	&	&55	 &45	&	&	&	& 100\\%\hline
WP-5.2	&	&	&	&	&	&	&	&	 &	&50	&60	&90	& 200\\\hline
Sum	& 200	&200	&200	&200	&200	&200	&200	&200	 &200	&200	&200	&200	& 2400\\\hline
\end{tabular}
\end{footnotesize}
\caption{Occupation (1PM) subject to the quarterly period and project task.}
\label{tableGesamtueberblick}
\end{table}


\paragraph*{WP-1: Review of recent advances in sentiment recognition}
\begin{tabular}{rl}
 Estimated effort& 6 PM\\
 Precondition & --\\
 Milestone & \begin{minipage}[t]{12.2cm}
\textit{Follow-up related work and recent developments}\vspace{.2cm}
             \end{minipage}
\end{tabular}

\noindent
In this workpackage we will follow up recent developments focusing sentiment sensing and opinion mining.
This ongoing task is distributed over the whole project duration.

\paragraph*{WP-2: Development of sentiment sensing systems}

\paragraph*{WP-2.1: Development of the basic recognition system}
\begin{tabular}{rl}
 Estimated effort& 3 PMs\\
 Precondition & --\\
 Milestone & \begin{minipage}[t]{12.2cm}
\textit{Sentiment sensing system with interfaces for various sensor types}\vspace{.2cm}
             \end{minipage}
\end{tabular}

\noindent
In this workpackage we develop a sensing system to attach medical, body, smartphone and environmental sensors.
The system will comprise interfaces for various sensing sources and cover of mobile clients to collect data samples and a server backend where the data is collected and processed. 
We perform preliminary analysis of the data collected from various sources to evaluate various aspects of them, such as, robustness, noise, and redundancy.
% \kobyc{add applciation for both collecting sensors and integrating them}

\paragraph*{WP-2.2: Integration of medical sensing equipment}
\begin{tabular}{rl}
	Estimated effort& 3 PMs\\
	Precondition & WP-2.1\\
	Milestone & \begin{minipage}[t]{12.2cm}
		\textit{Integration of medical sensor equipment to the  sensing system}\vspace{.2cm}
	\end{minipage}
\end{tabular}

\noindent
In this workpackage we integrate medical sensors (EMG and EOG) with the developed framework. 
We will have access to the medical sensors via a cooperation with the medical center at Georg-August University Goettingen, in particular with the Neurohabilitation Engineering Group of Professor Dario Farina\footnote{http://www.nre.bccn.uni-goettingen.de/index.php?id=82}. 
We will develop methods to predict the output of the medical sensors from smartphone and other sensors in WP-4.2 for the use in our sentiment prediction system.

\paragraph*{WP-2.3: Integration of smartphone and body sensors}
\begin{tabular}{rl}
 Estimated effort& 3 PMs\\
 Precondition & WP-2.1\\
 Milestone & \begin{minipage}[t]{12.2cm}
\textit{Integration of smartphone and body sensors to the sentiment sensing system}\vspace{.2cm}
             \end{minipage}
\end{tabular}

\noindent
In this workpackage we integrate smartphone and body sensors with the developed framework. 
Our focus will be on the integration of acceleration data from smartphones and sensor nodes. 
In particular, we will utilize INGA\footnote{https://www.ibr.cs.tu-bs.de/projects/inga/} sensor nodes to which we have access through a cooperation with TU Braunschweig, Germany. 
In addition, heart rate and accelerometer sensors, for instance, from fitness wristbands will be incorporated. 
We will evaluate the amount of sensor information such that we will be able to minimize usage of the phone's resource, and extend battery life.

% \kobyc{compression? + heart rate sensors? fitness wristband}

\paragraph*{WP-2.4: Integration of environmental sensors}
\begin{tabular}{rl}
 Estimated effort& 3 PMs\\
 Precondition & WP-2.1\\
 Milestone & \begin{minipage}[t]{12.2cm}
\textit{Integration of USRP software radio nodes to the  sensing system}\vspace{.2cm}
             \end{minipage}
\end{tabular}

\noindent
In this workpackage we integrate USRP\footnote{http://www.ettus.com} devices with the developed framework. 
For this, we can utilize 8 fully equipped USRP-1 devices and transceivers for the 800 MHz range available at the chair for Computer networks at Georg-August U. Goettingen.
% \kobyc{keep it?}

% \kobyc{I moved the original WP-2.1 "Integration of medical sensing equipment" to W-2.4 just below. I propose to replace it with a new workpackage just after it. This is because it will make our dependency in others smaller, and it will make the focus on something can be done everwhere.	}

\paragraph*{WP-2.5: Development of a mobile sensing system for continuous sampling}
\begin{tabular}{rl}
 Estimated effort& 3 PMs\\
 Precondition & WP-2.1,WP-2.2,WP-2.3,WP-2.4\\
 Milestone & \begin{minipage}[t]{12.2cm}
\textit{Android mobile sensing application}\vspace{.2cm}
             \end{minipage}
\end{tabular}

\noindent
An application for android mobile devices will be developed for mobile data collection. 
The application will incorporate smartphone sensors and enable the addition of further body sensors, for instance, via bluetooth.
In addition to the collection of data, the application will enable feedback to the user based on the sensed patterns. 
For instance, possible feedback would be recommendations conditioned on sensed sentiment.  
This application will be instrumented in WP-3.2.
The purpose of this application is the study of sentiment and the potential of user feedback to alter sentiment related behavior over a longer period of time. 
In particular, correlations between distinct sensor types, exploited in WP-4, will be utilized to mitigate missing sensor readings (e.g. from medical sensors) with sensor readings from other sensors.  

% \kobyc{can we combine both 3.1 and 3.2? what are the difference between them?}
\paragraph*{WP-3: Experiments}
\paragraph*{WP-3.1: Generation of experimental data samples}
\begin{tabular}{rl}
 Estimated effort& 6 PMs\\
 Precondition & WP-2\\
 Milestone & \begin{minipage}[t]{12.2cm}
\textit{Generation of a data set for sentiment analysis}\vspace{.2cm}
             \end{minipage}
\end{tabular}

% \kobyc{edit the following if we modify wp 2.4 above}
\noindent
Utilizing the sentiment sensing system developed in WP-2, we will collect data from medical, smartphone, body and environmental sensors for our sentiment analysis in WP-4. 
We will consider two classes of medical sensors, for instance, EMG and EOG, acceleration data from the INGA and smartphone sensors as well as RF-signal information captured by USRP devices. 
We aim at collecting a body of data from 30-50 subjects in laboratory settings in which a series of standard tasks will be conducted by the subjects while the sentiment classes are induced by the experiment design. 
The subjects will be recruited from students and University staff as well as from patients or medical test persons in cooperation with the medical center at University of Goettingen.
% We will conduct two separate data sets over the course of the project in two separate three-month periods. 
At least three sentiment classes will be considered, such as happiness, stress, and tiredness. 
The experimental setting will be designed to induce these sentiment classes in role-play and interactive games.
To generate ground truth, participants we will collect feedback after the experiments. 
In the test design and execution we will receive support from the Institute of Psychology at TU Braunschweig, let by Professor Simone Kauffeld\footnote{https://www.tu-braunschweig.de/psychologie/abt/aos/mitarbeiterinnen/kauffeld/index.html}.
The purpose of this workpackage is the generation of data for the development of accurate classifiers for sentiment prediction given all available sensor modalities and the identification of possible correlations between sensor classes.

\paragraph*{WP-3.2: Mobile data collection from smartphones}
\begin{tabular}{rl}
 Estimated effort& 6 PMs\\
 Precondition & WP-2\\
 Milestone & \begin{minipage}[t]{12.2cm}
\textit{Generation of a data set for the prediction of sentiment}\vspace{.2cm}
             \end{minipage}
\end{tabular}

\noindent
We will collect acceleration data from smartphone and body sensors for our sentiment prediction considered in WP-5. 
In this experiment, the accelerometer data from 10-20 persons will be collected over a period of 6 months both in Israel and Germany. 
These individuals, recruited from students and staff of both universities, will be equipped with body sensors measuring acceleration data as well as with portable medical sensors. 
In order to reach dense sampling, we will consider the use of fitness trackers\footnote{for instance https://www.fitbit.com/de/chargehr} and monetary reward systems. 
During the experiments, in order to improve our confidence on the ground truth, users will occasionally be asked for feedback by the application.
In addition, users will receive recommendations based on the sensed sentiment patterns. 

% \kobyc{both in germany and israel}
% \kobyc{ask the user for feedback WP-2.2}

\paragraph*{WP-3.3: User interaction and feedback}
\begin{tabular}{rl}
 Estimated effort& 3 PMs\\
 Precondition & WP-3.2\\
 Milestone & \begin{minipage}[t]{12.2cm}
\textit{Report on the efficiency of the installed feedback mechanisms}\vspace{.2cm}
             \end{minipage}
\end{tabular}

\noindent
In this workpackage, the efficiency and performance of the implemented feedback mechanisms of the app implemented in WP-3.2.
In particular, this feedback is generated by user questionnaires.
In the questionnaire design and execution we will receive support from the Institute of Psychology at TU Braunschweig, let by Professor Simone Kauffeld\footnote{https://www.tu-braunschweig.de/psychologie/abt/aos/mitarbeiterinnen/kauffeld/index.html}.
Two feedback rounds are implemented in the middle and at the end of WP-3.2 in order to enable adaptation of the feedback in the second phase. 

\paragraph*{WP-4: Sentiment analysis}
% \kobyc{maybe remove 4.1 or integrate it with 4.2?}
\paragraph*{WP-4.1: Sentiment analysis from medical sensors}
\begin{tabular}{rl}
 Estimated effort& 3 PMs\\
 Precondition & WP-3\\
 Milestone & \begin{minipage}[t]{12.2cm}
\textit{Suitable features and a classifier for sentiment from at least two types of medical sensors}\vspace{.2cm}
             \end{minipage}
\end{tabular}

\noindent
In this workpackage we investigate the identification of sentiment from data of medical sensors. 
We will develop new learning methods for detecting sentiment from EMG and EOG data. 
These methods will take into consideration the special nature of these signals.

\paragraph*{WP-4.2: Sentiment analysis from body and smartphone sensors}
\begin{tabular}{rl}
 Estimated effort& 6 PMs\\
 Precondition & WP-3, WP-4.1\\
 Milestone & \begin{minipage}[t]{12.2cm}
\textit{Correlations between features from medical and smartphone sensors and a classifier for sentiment from these sensors}\vspace{.2cm}
             \end{minipage}
\end{tabular}

\noindent
We will in this workpackage develop suitable features for the prediction of sentiment from body and smartphone sensors. Our goal is to have a small number of predictive features. 
In particular, similarly to previous work regarding movement, gestures and pose as indicators of emotion, classifiers for the detection of such classes will be developed and linked to the classification of sentiment. 
In addition, building on the results from WP-4.1 we will further investigate correlations between sensor readings of medical and acceleration sensors in order to give an estimation to which extent medical sensors can be substituted by cheaper acceleration sensors for the application in large scale on-phone sensing applications.

\paragraph*{WP-4.3: Sentiment classification from body and smartphone sensors}
\begin{tabular}{rl}
 Estimated effort& 6 PMs\\
 Precondition & WP-3, WP-4.1, WP-4.2\\
 Milestone & \begin{minipage}[t]{12.2cm}
\textit{Classifier for sentiment over extended period of time}\vspace{.2cm}
             \end{minipage}
\end{tabular}

\noindent
In this workpackage, we focus the analysis of sentiment over an extended period of time. 
In particular, we investigate the identification of typical sentiment patterns from the observed sensor data.
From this, we focus on the prediction of sentiment based not only on current sensor input, but also on recent historical data. 
That is, modelling the state-of-mind of a person, and use it to improve prediction of future state-of-mind. 
In particular, similar to our previous work~\cite{4026,4027}, we consider the use of alignment matching approaches to identify approximately similar sub-patterns in sentiment time series and to predict probable continuation of these patterns.

% \kobyc{related sentiment to location or time?}

\paragraph*{WP-4.4: Sentiment analysis from environmental sensors}
\begin{tabular}{rl}
 Estimated effort& 6 PMs\\
 Precondition & WP-3\\
 Milestone & \begin{minipage}[t]{12.2cm}
\textit{Features and a classifier for sentiment from received RF-signals}\vspace{.2cm}
             \end{minipage}
\end{tabular}

\noindent
We will in this workpackage develop features for the prediction of sentiment from received RF signals. 
Building on our and other previous work detecting movement and gestures from received RF-signals, we will detect gestures, movement and pose of individuals. 
Then, we will devise new methods to predict from the output of these prediction regarding physical state (movement, gestures and pose) mental state, such as emotion and sentiment. 

\paragraph*{WP-4.5: Sensor output analysis and fusion}
\begin{tabular}{rl}
	Estimated effort& 3 PMs\\
	Precondition & WP-2.1,WP-2.2,WP-2.3\\
	Milestone & \begin{minipage}[t]{12.2cm}
		\textit{Analysis and fusion of output from all sensors.}\vspace{.2cm}
	\end{minipage}
\end{tabular}

\noindent
In this workpackage we will analyse the sensor data and evaluate their redundancy. This study will be used to develop methods to fuze data from all sensors into a single coherent and compact stream. We will develop methods to remove noise and outliers readings. These tools will be used to process sensor data before feeding it into sentiment prediction models.


\paragraph*{WP-5: Prediction of sentiment based on past data and provide feedback for future}
\paragraph*{WP-5.1: Integrating past sentiment}
\begin{tabular}{rl}
 Estimated effort& 3 PMs\\
 Precondition & WP-3, WP-4\\
 Milestone & \begin{minipage}[t]{12.2cm}
\textit{A document describing the potential of sentiment analysis of long sensor traces}\vspace{.2cm}
             \end{minipage}
\end{tabular}

\noindent
We will analyse long sequences of sensor-input and sentimental-state. Our goal is to find long-correlations between the state of the sentiment across in various time scales (minutes, hours, days). Based on these results we will develop models to predict sentiment based on both current sensor data and previous (or historical) sentiment, either predicted (or also given via interface). 


\paragraph*{WP-5.2: Generating user feedback}
\begin{tabular}{rl}
	Estimated effort& 3 PMs\\
	Precondition & WP-5-1\\
	Milestone & \begin{minipage}[t]{12.2cm}
		\textit{A document describing the potential of sentiment prediction from long sensor traces}\vspace{.2cm}
	\end{minipage}
\end{tabular}

\noindent
We will develop few feedback methods to users about current and future predicted sentiment. Our goal is to build an automatic system that will find an optimal feedback mechanism to achieve certain goals, defined by the user. We plan to build on recent advances~\cite{DBLP:journals/ml/CrammerG13} in multi-armed bandit algorithms based on context which are optimizing exploration of methods and exploiting them.


\subsection*{Type and extent of the cooperation and division of tasks between the partners}
Goettingen's expertise lie in mobile pervasive sensing systems, distributed, networked systems and compressive sensing.  
Technion's expertise lie in the design of new methods for data analysis, building predictive models, and devise highly efficient methods. We envision a relation in which Goettingen take the lead on sensor deployment and analysis, and Technion on analysis and building prediction models. We plan to close the loop, and use the models and analysis-outcome to better improve sensor usage. Smartphones and body sensors will be deployed on both locations. We will to assign a student in each institute, and plan for them to work as a team.

The researchers plan to have periodic phone meetings, and visit each other every six months. We hope that our Goettingen's students will be have long visit in Technion in vice-versa. This will not only advance the project, but will also serve, as a way to deepen their network and the connections between the two institutes. 


\subsection*{Anticipated results}
This project will generate a body of knowledge on the recognition of sentiment from various sensor types.
In particular, in the scope of the project, a prototype implementation for a mobile sentiment recognition app is developed and a corpus of sensor readings together with the associated ground truth is provided. 
The results will provide insight into the potential of sentiment sensing and prediction from medical, smartphone and body sensors.

\subsection*{Relevant preceding work of the applicants}
% \begin{itemize}
%  \item[\color{blue}tbd]
% \end{itemize}
\subsubsection*{Prediction of sensor readings}
Stephan Sigg has been working on the prediction/continuation of time series from sensor readings between 2005-2009.
In this analytic work he showed that the order in which processing operations are applied to sensor readings impacts the overall error probability for the prediction algorithm, regardless of the actual prediction approach utilized.
A stochastic model to estimate this impact has been presented and evaluated in simulations and case studies~\cite{4027}. 
Also, a novel class of context-prediction methods based on alignment approaches~\cite{4026} was derived which was shown to outperform traditional approaches on nominal data sequences. 
This approach has been re-used by various groups for the prediction of contextual information.
Its complexity has been derived analytically and was compared to alternative context prediction methods~\cite{4027}. 
The method was applied to readings of inertial sensors, GPS readings and long-time weather data. 

\subsubsection*{Recognition of human activities from on-body and RF sensors}
Since 2009 Stephan Sigg considers the use of fluctuations measured on the Radio Frequency (RF) channel for the recognition of activities and gestures.
The derived systems exploit multi-path propagation as well as the reflection and blocking of electromagnetic signals.
Movement is reflected in the signal strength and fluctuation patterns of a received signal.
One benefit of such device-free approaches is that it is not necessary for individuals to actually wear a transmit or receive device.
It was shown that the localisation and recognition of simple activities is possible from continuous signals whereas the accuracy is comparable to recognition systems that rely on body-worn accelerometers~\cite{Pervasive_Sigg_2012}. 
These considerations have been extended to passive systems which utilize ambient signals~\cite{Pervasive_Sigg_2013}.
Recent results show, that it is further possible to detect directed attention from fluctuations on the RF-channel by interpreting body movement captured from RF-readings~\cite{Pervasive_Shi_2014}.
Such gestures can be detected based on WiFi RSSI using only time-domain features on an off-the-shelf mobile phone~\cite{RFSensing_Sigg_2014}.

\subsubsection*{Developing Prediction Models}
Koby Crammer have been working on developing prediction models since his PhD. He developed new methods for assigning one category~\cite{CrammerSi00,CrammerSi00a,CrammerSi01,CrammerSi01a,CrammerDrKu09,DBLP:journals/jmlr/LivniCG12,DBLP:conf/icml/WangCV10,DBLP:conf/ijcai/VatashskyC13,DBLP:journals/jmlr/WangCV12}
 or more~\cite{CrammerSi03,DBLP:conf/naacl/McDonaldCP05} to a given input. The task of predicting sentiment (or composed sentimental state) from sensor input share some properties with these problems. Another related line of research pursued by Koby is predicting sentiment about product from human reviews~\cite{DBLP:journals/jmlr/HaimovitchCM12,dredze-ml-09,DBLP:journals/ml/CrammerKD13}. 
  
 Koby also worked extensively on developing new learning methods and models to predict sequential data and taking correlations into account. These methods have been used in text analysis~\cite{mcdonald-04,mejer2010confidence}, speech analysis~\cite{DBLP:conf/icassp/CrammerL12,DBLP:conf/icassp/Crammer10} and bioinformatics~\cite{DBLP:journals/bioinformatics/BernalCP12,DBLP:journals/ploscb/BernalCHP07,DBLP:journals/bmcbi/LiuCPR08}.


\subsubsection*{Developing Efficient Algorithms}
Since 2008 Koby have worked largely on new adaptive models and algorithms that are very time and memory efficient. This worked is based on earlier work~\cite{DBLP:journals/jmlr/CrammerDKSS06} on online learning. The methods were designed for predicting a real number~\cite{CrammerKuDr12,MoroshkoC12}, single bit from an input~\cite{DBLP:journals/ml/CrammerKD13,Crammer:2012:CLC:2343676.2343704,DBLP:conf/nips/CrammerL10}, one out for few~\cite{CrammerDrKu09,dredze-ml-09}, or many decisions simultaneously~\cite{mejer2012training,DBLP:conf/icassp/CrammerL12}. Additional work pursed are new efficient methods for large scale learning~\cite{DBLP:conf/icml/WangCV10,DBLP:journals/jmlr/WangCV12}.

Koby have also developed new methods that are efficient in interaction with annotation agents, and specifically human annotators. These methods can require annotation only for parts of the input~\cite{DBLP:journals/jmlr/HaimovitchCM12,DBLP:conf/pkdd/OrbachC12,OrbachCr12ieeei}, or they actively choose what inputs should be annotated~\cite{dredze-08b,DBLP:journals/ml/CrammerG13,mejer2012training}.

\subsubsection*{Detecting outliers}
Koby Crammer developed few models to detect outliers in data or finding a small subset of measurements that can be representative and coherent. The methods are either working by enclosing a large subset of inputs in a small ball~\cite{DBLP:conf/colt/CrammerS03}, using non-convex loss functions~\cite{CrammerCh04}, or rate-distortion methods from information theory~\cite{DBLP:conf/icml/CrammerTP08}.

\subsection*{Brief details of the type and the extent of previous cooperation between applicants}
The applicants were introduced to each other by a third researcher. They communicated tightly and continuously for both setting the project goal and methods, and, clearly, writing this application. They plan to collaborate anyway, and this grant will serve as an appropriate platform for a fruitful joint work.

\subsection*{Perspectives with respect to possible follow-up projects between the partners with funding from other sources}
Depending on the potential and results of the present project, we are considering to request funding in the frame of the DFG German-Israeli Project Cooperation (DIP)\footnote{http://www.dfg.de/en/research\_funding/programmes/international\_cooperation/german\_israeli\_cooperation/index.html} or in the frame of a H2020 FET Open project\footnote{http://ec.europa.eu/programmes/horizon2020/en/h2020-section/fet-open}
A further alternative, involving younger scientists, are the establishing of Initial Training Networks\footnote{http://ec.europa.eu/research/mariecurieactions/about-mca/actions/itn/index\_en.htm}.

\subsection*{Funding of the project including own contribution}
The project teams consists of the project researchers Dr. Koby Crammer and Dr. Stephan Sigg as well as two PhD students (to be appointed for this project) and assistants.
\begin{description}
	\nolineskips
	\item[UGOE] Dr. Stephan Sigg (externally funded), 3 years
	\item[UGOE] 1 research staff member, TV-L 13, 74\% project + 26\% external, 3 years
	\item[UGOE] 1 student assistant, 40 hours/week (externally funded) 3 years
	\item[Technion] Dr. Koby Crammer (externally funded), 3 years 
	\item[Technion] 1 research staff member, PhD student for three years
	\item[Technion] 1 postdoctoral fellow, 50\% project + 50\% externally, 3 years
	\item[Technion] 1 IT engineer, 10\% project + 90\% externally, 3 years
\end{description}
Technion will make use of the machine learning computing cluster composed of more than 50 nodes. If there will be a need, Technion will make the proper adaptation of the cluster (storage) to the possible requirements of the project. Additionally, some of the funds will be used to purchase the necessary hardware for the project, such as smartphones and USRPs.


\pagebreak

\bibliographystyle{plain}
% argument is your BibTeX string definitions and bibliography database(s)
%\bibliography{IEEEabrv,../bib/paper}
%
% <OR> manually copy in the resultant .bbl file
% set second argument of \begin to the number of references
% (used to reserve space for the reference number labels box)
% \begin{thebibliography}{1}
% 
% \bibitem{IEEEhowto:kopka}
% H.~Kopka and P.~W. Daly, \emph{A Guide to \LaTeX}, 3rd~ed.\hskip 1em plus
%   0.5em minus 0.4em\relax Harlow, England: Addison-Wesley, 1999.
% 
% \end{thebibliography}
\begin{small}
\begin{thebibliography}{10}

\bibitem{RFSensing_Kellog_2014}
Bringing gesture recognition to all devices.
\newblock In {\em Presented as part of the 11th USENIX Symposium on Networked
  Systems Design and Implementation}, Berkeley, CA, 2014. USENIX.

\bibitem{RFSensing_Adib_2014}
F.~Adib, Z.~Kabelac, D.~Katabi, and R.~C. Miller.
\newblock 3d tracking via body radio reflections.
\newblock available online: http://witrack.csail.mit.edu/witrack-paper.pdf,
  2014.
\newblock Usenix NSDI’14.

\bibitem{Pervasive_Adib_2013}
Fadel Adib and Dina Katabi.
\newblock See through walls with wi-fi.
\newblock In {\em ACM SIGCOMM'13}, 2013.

\bibitem{Emotion_Nguyen_2012}
Matthias~Rauterberg Anh-Tuan~Nguyen, Wei~Chen.
\newblock The role of human body expression in affect detection: A review.
\newblock In {\em 10th Asia Pacific Conference on Computer Human Interaction
  (APCHI 2012)}, 2012.

\bibitem{Emotion_Atkinson_2007}
Anthony~P Atkinson, Mary~L Tunstall, and Winand~H Dittrich.
\newblock Evidence for distinct contributions of form and motion information to
  the recognition of emotions from body gestures.
\newblock {\em Cognition}, 104(1):59--72, 2007.

\bibitem{Pervasive_Bao_2004}
L.~Bao and S.~S. Intille.
\newblock Activity recognition from user-annotated acceleration data.
\newblock In {\em Proceedings of PERVASIVE 2004}, volume LNCS 3001, 2004.

\bibitem{Emotion_Bernhardt_2007}
Daniel Bernhardt and Peter Robinson.
\newblock Detecting affect from non-stylised body motions.
\newblock In {\em Affective Computing and Intelligent Interaction}, pages
  59--70. Springer, 2007.

\bibitem{Emotion_Bull_1987}
Peter~E Bull.
\newblock {\em Posture and gesture.}
\newblock Pergamon press, 1987.

\bibitem{SentimentSensing_Calvo_2010}
Rafael~A Calvo and Sidney D'Mello.
\newblock Affect detection: An interdisciplinary review of models, methods, and
  their applications.
\newblock {\em Affective Computing, IEEE Transactions on}, 1(1):18--37, 2010.

\bibitem{SentimentSensing_Cambria_2013}
Erik Cambria, Bjorn Schuller, Yunqing Xia, and Catherine Havasi.
\newblock New avenues in opinion mining and sentiment analysis.
\newblock {\em IEEE Intelligent Systems}, page~1, 2013.

\bibitem{Pervasive_Cao_2012}
Hong Cao, Minh~Nhut Nguyen, Clifton Chun~Wei Phua, Shonali Krishnaswamy, and
  Xiaoli Li.
\newblock An integrated framework for human activity classification.
\newblock In {\em Proceedings of the 14th ACM International Conference on
  Ubiquitous Computing (UbiComp 2012)}, 2012.

\bibitem{Emotion_Castellano_2008}
Ginevra Castellano, Loic Kessous, and George Caridakis.
\newblock Emotion recognition through multiple modalities: face, body gesture,
  speech.
\newblock In {\em Affect and emotion in human-computer interaction}, pages
  92--103. Springer, 2008.

\bibitem{Pevasive_Chavarriaga_2011}
Ricardo Chavarriaga, Hamidreza Bayati, and Jose del R.~Millan.
\newblock Unsupervised adaptation for acceleration-based activity recognition:
  robustness to sensor displacement and rotation.
\newblock {\em Personal and Ubiquitous Computing}, 2011.

\bibitem{Pervasive_Cohn_2012}
Gabe Cohn, Dan Morris, Shwetak~N. Patel, and Desney~S. Tan.
\newblock Humantenna: Using the body as an antenna for real-time whole-body
  interaction.
\newblock In {\em Proceedings of ACM CHI 2012}, 2012.

\bibitem{Emotion_Crane_2007}
Elizabeth Crane and Melissa Gross.
\newblock Motion capture and emotion: Affect detection in whole body movement.
\newblock In {\em Affective computing and intelligent interaction}, pages
  95--101. Springer, 2007.

\bibitem{Emotion_DeMeijer_1989}
Marco De~Meijer.
\newblock The contribution of general features of body movement to the
  attribution of emotions.
\newblock {\em Journal of Nonverbal behavior}, 13(4):247--268, 1989.

\bibitem{Pervasive_Dey_2011}
Anind~Kumar Dey, K.~Wac, D.~Ferreira, K.~Tassini, J.-H. Hong, and J.~Ramos.
\newblock Getting closer: An empirical investigation of the proximity of user
  to their smart phones.
\newblock In {\em Proceedings of the 13th international conference on
  Ubiquitous computing}, 2011.

\bibitem{Emotion_Dittmann_1978}
Allen~T Dittmann.
\newblock The role of body movement in communication.
\newblock {\em Nonverbal behavior and communication}, pages 69--95, 1978.

\bibitem{SentimentSesing_Mello_2009}
Sidney D'Mello and Art Graesser.
\newblock Automatic detection of learner's affect from gross body language.
\newblock {\em Applied Artificial Intelligence}, 23(2):123--150, 2009.

\bibitem{SentimentSensing_Ekman_2003}
Paul Ekman and Wallace~V Friesen.
\newblock {\em Unmasking the face: A guide to recognizing emotions from facial
  clues}.
\newblock Ishk, 2003.

\bibitem{AttentionMonitoring_Ferscha_2012}
Alois Ferscha, Kashif Zia, and Benedikt Gollan.
\newblock Collective attention through public displays.
\newblock In {\em 2012 IEEE Sixth International Conference on Self-Adaptive and
  Self-Organizing Systems (SASO)}, pages 211 --216, 2012.

\bibitem{AttentionMonitoring_Gollan_2011}
Benedikt Gollan, B.~Wally, and Alois Ferscha.
\newblock Automatic attention estimation in an interactive system based on
  behaviour analysis.
\newblock In {\em Proceedings of the 15th Portuguese Conference on Artificial
  Intelligence (EPIA2011)}, 2011.

\bibitem{SentimentSensing_Gross_1995}
James~J Gross and Ricardo~F Mu{\~n}oz.
\newblock Emotion regulation and mental health.
\newblock {\em Clinical psychology: Science and practice}, 2(2):151--164, 1995.

\bibitem{Pervasive_Jaggarwal_2012}
Akhil Jaggarwal and Roxanne~L Canosa.
\newblock Emotion recognition using body gesture and pose.
\newblock Technical report, 2012.
\newblock available online:
  http://www.cs.rit.edu/$\sim$axj4159/papers\_march/report\_1.pdf.

\bibitem{SentimentSensing_Juslin_2005}
Patrik~N Juslin and Klaus~R Scherer.
\newblock {\em Vocal expression of affect.}
\newblock Oxford University Press, 2005.

\bibitem{RFsensing_Kim_2009}
Youngwook Kim and Hao Ling.
\newblock Human activity classification based on micro-doppler signatures using
  a support vector machine.
\newblock {\em IEEE Transactions on Geoscience and Remote Sensing},
  47(5):1328--1337, 2009.

\bibitem{SentimentSensing_Kramer_2014}
Adam~DI Kramer, Jamie~E Guillory, and Jeffrey~T Hancock.
\newblock Experimental evidence of massive-scale emotional contagion through
  social networks.
\newblock {\em Proceedings of the National Academy of Sciences}, page
  201320040, 2014.

\bibitem{Emotion_Lagerlof_2009}
Ingrid Lagerl{\"o}f and Marie Djerf.
\newblock Children's understanding of emotion in dance.
\newblock {\em European Journal of Developmental Psychology}, 6(4):409--431,
  2009.

\bibitem{Pervasive_Lane_2010}
Nicholas~D. Lane, Emiliano Miluzzo, Hong Lu, Daniel Peebles, Tanzeem Choudhury,
  and Andrew~T. Campbell.
\newblock A survey of mobile phone sensing.
\newblock {\em IEEE Communications magazine}, 48(9):140--150, September 2010.

\bibitem{SentimentSensing_Liu_2012}
Bing Liu and Lei Zhang.
\newblock A survey of opinion mining and sentiment analysis.
\newblock In {\em Mining text data}, pages 415--463. Springer, 2012.

\bibitem{SentimentSensing_Lupien_2009}
Sonia~J Lupien, Bruce~S McEwen, Megan~R Gunnar, and Christine Heim.
\newblock Effects of stress throughout the lifespan on the brain, behaviour and
  cognition.
\newblock {\em Nature Reviews Neuroscience}, 10(6):434--445, 2009.

\bibitem{SentimentSensing_Lyubomirsky_2005}
Sonja Lyubomirsky, Laura King, and Ed~Diener.
\newblock The benefits of frequent positive affect: does happiness lead to
  success?
\newblock {\em Psychological bulletin}, 131(6):803, 2005.

\bibitem{Emotion_Meeren_2005}
Hanneke~KM Meeren, Corn{\'e}~CRJ van Heijnsbergen, and Beatrice de~Gelder.
\newblock Rapid perceptual integration of facial expression and emotional body
  language.
\newblock {\em Proceedings of the National Academy of Sciences of the United
  States of America}, 102(45):16518--16523, 2005.

\bibitem{Emotion_Montepare_1999}
Joann Montepare, Elissa Koff, Deborah Zaitchik, and Marilyn Albert.
\newblock The use of body movements and gestures as cues to emotions in younger
  and older adults.
\newblock {\em Journal of Nonverbal Behavior}, 23(2):133--152, 1999.

\bibitem{SentimentSensing_Mota_2003}
Selene Mota and Rosalind~W Picard.
\newblock Automated posture analysis for detecting learner's interest level.
\newblock In {\em Computer Vision and Pattern Recognition Workshop, 2003.
  CVPRW'03. Conference on}, volume~5, pages 49--49. IEEE, 2003.

\bibitem{SentimentSensing_Pang_2008}
Bo~Pang and Lillian Lee.
\newblock Opinion mining and sentiment analysis.
\newblock {\em Foundations and trends in information retrieval}, 2(1-2):1--135,
  2008.

\bibitem{Pervasive_Patel_2006}
Shwetak~N. Patel, J.~A. Kientz, G.~R. Hayes, S.~Bhat, and Gregory~D. Abowd.
\newblock Farther than you may think: An empirical investigation of the
  proximity of users to their mobile phones.
\newblock In {\em Proceedings of the 8th international conference on Ubiquitous
  computing}, pages 123--140, 2006.

\bibitem{Pervasive_Patridge_2008}
K.~Patridge and P.~Golle.
\newblock On using existing time-use study data for ubiquitous computing
  applications.
\newblock In {\em Proceedings of the 10th International Conference on
  Ubiquitous Computing}, pages 144--153, 2008.

\bibitem{Pervasive_Ploetz_2012}
Thomas Ploetz, Nils~Y. Hammerla, Agata Rozga, Andrea Reavis, Nathan Call, and
  Gregory~D. Abowd.
\newblock Automatic assessment of problem behavior in individuals with
  developmental disabilities.
\newblock In {\em Proceedings of the 14th ACM International Conference on
  Ubiquitous Computing (Ubicomp 2012)}, 2012.

\bibitem{RFsensing_Pu_2013}
Qifan Pu, Sidhant Gupta, Shyam Gollakota, and Shwetak Patel.
\newblock Whole-home gesture recognition using wireless signals.
\newblock In {\em The 19th Annual International Conference on Mobile Computing
  and Networking (Mobicom'13)}, 2013.

\bibitem{Pervasive_Ravi_2005}
Nishkam Ravi, Nikhil Dandekar, Preetham Mysore, and Michael~L. Littman.
\newblock Activity recognition from accelerometer data.
\newblock In {\em Proceedings of the 17th conference on Innovative applications
  of artificial intelligence - Volume 3}, IAAI'05, pages 1541--1546, 2005.

\bibitem{SentimentSensing_Salovey_2000}
Peter Salovey, Alexander~J Rothman, Jerusha~B Detweiler, and Wayne~T Steward.
\newblock Emotional states and physical health.
\newblock {\em American psychologist}, 55(1):110, 2000.

\bibitem{Pervasive_Shi_2014}
Shuyu Shi, Stephan Sigg, Wei Zhao, and Yusheng Ji.
\newblock Monitoring of attention from ambient fm-radio signals.
\newblock {\em IEEE Pervasive Computing, Special Issue - Managing Attention in
  Pervasive Environments}, Jan-March 2014.

\bibitem{RFSensing_Sigg_2014}
Stephan Sigg, Ulf Blanke, and Gerhard Troester.
\newblock The telepathic phone: Frictionless activity recognition from
  wifi-rssi.
\newblock In {\em IEEE International Conference on Pervasive Computing and
  Communications (PerCom)}, PerCom '14, 2014.

\bibitem{4027}
Stephan Sigg, Dawud Gordon, Georg~von Zengen, Michael Beigl, Sandra Haseloff,
  and Klaus David.
\newblock Investigation of context prediction accuracy for different context
  abstraction levels.
\newblock {\em IEEE Transactions on Mobile Computing}, 11(6):1047 --1059, june
  2012.

\bibitem{4026}
Stephan Sigg, Sandra Haseloff, and Klaus David.
\newblock An alignment approach for context prediction tasks in ubicomp
  environments.
\newblock {\em IEEE Pervasive Computing}, Oct-Dec 2010, 2010.

\bibitem{Pervasive_Sigg_2012}
Stephan Sigg, Markus Scholz, Shuyu Shi, Yusheng Ji, and Michael Beigl.
\newblock Rf-sensing of activities from non-cooperative subjects in device-free
  recognition systems using ambient and local signals.
\newblock {\em IEEE Transactions on Mobile Computing}, 13(4), 2013.

\bibitem{Pervasive_Sigg_2013}
Stephan Sigg, Shuyu Shi, Felix Buesching, Yusheng Ji, and Lars Wolf.
\newblock Leveraging rf-channel fluctuation for activity recognition.
\newblock In {\em Proceedings of the 11th International Conference on Advances
  in Mobile Computing and Multimedia (MoMM2013)}, 2013.

\bibitem{SentimentSensing_Smith_2004}
Timothy~W Smith, Kelly Glazer, John~M Ruiz, and Linda~C Gallo.
\newblock Hostility, anger, aggressiveness, and coronary heart disease: An
  interpersonal perspective on personality, emotion, and health.
\newblock {\em Journal of personality}, 72(6):1217--1270, 2004.

\bibitem{SentimentSensing_Todaro_2003}
John~F Todaro, Biing-Jiun Shen, Raymond Niaura, Avron Spiro, and Kenneth~D
  Ward.
\newblock Effect of negative emotions on frequency of coronary heart disease
  (the normative aging study).
\newblock {\em The American journal of cardiology}, 92(8):901--906, 2003.

\bibitem{Emotion_VanHeijnsbergen_2007}
CCRJ Van~Heijnsbergen, HKM Meeren, J~Grezes, and B~de~Gelder.
\newblock Rapid detection of fear in body expressions, an erp study.
\newblock {\em Brain research}, 1186:233--241, 2007.

\bibitem{Pervasive_Varshavsky_2008}
Alexander Varshavsky and Shwetak~N. Patel.
\newblock {\em Location in Ubiquitous Computing}, pages 285--319.
\newblock Ubiquitous Computing Fundamentals. Taylor and Francis Group, 2008.

\bibitem{Emotion_Wallbott_1998}
Harald~G Wallbott.
\newblock Bodily expression of emotion.
\newblock {\em European journal of social psychology}, 28(6):879--896, 1998.

\bibitem{Emotion_Walters_1986}
K.~Walters and R.~Walk.
\newblock Perception of emotion from body posture.
\newblock {\em Bulletin of the Psychonomic Society}, 24(5), 1986.

\bibitem{AttentionMonitoring_Wickens_1984}
C.D. Wickens.
\newblock {\em Processing resources in attention}.
\newblock Academic Press, 1984.

\bibitem{AttentionMonitoring_Wickens_2008}
C.D. Wickens and J.S. McCarley.
\newblock {\em Applied attention theory}.
\newblock CRC Press, 2008.

\bibitem{AttentionMonitoring_Wu_2007}
F.~Wu and B.A. Hubermann.
\newblock Novelty and collective attention.
\newblock In {\em Proceedings of the National Academics of Sciences}, volume
  104, pages 17599--17601, 2007.

\bibitem{AttentionMonitoring_Xu_2012}
Yongchun Xu, Nenad Stojanovic, Ljiljana Stojanovic, and Tobias Schuchert.
\newblock Efficient human attention detection based on intelligent complex
  event processing.
\newblock In {\em Proceedings of the 6th ACM International Conference on
  Distributed Event-Based Systems}, DEBS '12, pages 379--380, 2012.

\bibitem{AttentionMonitoring_Yonezawa_2007}
Tomoko Yonezawa, Hirotake Yamazoe, Akira Utsumi, and Shinji Abe.
\newblock Gaze-communicative behavior of stuffed-toy robot with joint attention
  and eye contact based on ambient gaze-tracking.
\newblock In {\em ICMI}, pages 140--145, 2007.

\bibitem{SentimentSensing_Zhai_2011}
Zhongwu Zhai, Hua Xu, Bada Kang, and Peifa Jia.
\newblock Exploiting effective features for chinese sentiment classification.
\newblock {\em Expert Systems with Applications}, 38(8):9139--9146, 2011.


% kkk

\bibitem{DBLP:journals/ploscb/BernalCHP07}
Axel Bernal, Koby Crammer, Artemis~G. Hatzigeorgiou, and Fernando Pereira.
\newblock Global discriminative learning for higher-accuracy computational gene
prediction.
\newblock {\em PLoS Computational Biology}, 3(3), 2007.

\bibitem{DBLP:journals/bioinformatics/BernalCP12}
Axel Bernal, Koby Crammer, and Fernando Pereira.
\newblock Automated gene-model curation using global discriminative learning.
\newblock {\em Bioinformatics}, 28(12):1571--1578, 2012.

\bibitem{CrammerCh04}
K.~Crammer and G.~Chechik.
\newblock A needle in a haystack: Local one-class optimization.
\newblock 2004.

\bibitem{CrammerDrKu09}
K.~Crammer, M.~Dredze, and A.~Kulesza.
\newblock Multi-class confidence weighted algorithms.
\newblock In {\em EMNLP}, 2009.

\bibitem{CrammerSi00a}
K.~Crammer and Y.~Singer.
\newblock Improved output coding for classification using continuous
relaxation.
\newblock 2000.

\bibitem{CrammerSi00}
K.~Crammer and Y.~Singer.
\newblock On the learnability and design of output codes for multiclass
problems.
\newblock 2000.

\bibitem{CrammerSi01a}
K.~Crammer and Y.~Singer.
\newblock On the algorithmic implementation of multiclass kernel-based vector
machines.
\newblock {\em Jornal of Machine Learning Research}, 2:265--292, 2001.

\bibitem{CrammerSi01}
K.~Crammer and Y.~Singer.
\newblock Ultraconservative online algorithms for multiclass problems.
\newblock 2001.

\bibitem{CrammerSi03}
K.~Crammer and Y.~Singer.
\newblock A new family of online algorithms for category ranking.
\newblock {\em Jornal of Machine Learning Research}, 3:1025--1058, 2003.

\bibitem{DBLP:conf/icassp/Crammer10}
Koby Crammer.
\newblock Efficient online learning with individual learning-rates for phoneme
sequence recognition.
\newblock In {\em Proceedings of the {IEEE} International Conference on
	Acoustics, Speech, and Signal Processing, {ICASSP} 2010, 14-19 March 2010,
	Sheraton Dallas Hotel, Dallas, Texas, {USA}}, pages 4878--4881, 2010.

\bibitem{DBLP:journals/jmlr/CrammerDKSS06}
Koby Crammer, Ofer Dekel, Joseph Keshet, Shai Shalev{-}Shwartz, and Yoram
Singer.
\newblock Online passive-aggressive algorithms.
\newblock {\em Journal of Machine Learning Research}, 7:551--585, 2006.

\bibitem{Crammer:2012:CLC:2343676.2343704}
Koby Crammer, Mark Dredze, and Fernando Pereira.
\newblock Confidence-weighted linear classification for text categorization.
\newblock {\em J. Mach. Learn. Res.}, 98888:1891--1926, June 2012.

\bibitem{DBLP:journals/ml/CrammerG13}
Koby Crammer and Claudio Gentile.
\newblock Multiclass classification with bandit feedback using adaptive
regularization.
\newblock {\em Machine Learning}, 90(3):347--383, 2013.

\bibitem{CrammerKuDr12}
Koby Crammer, Alex Kulesza, and Mark Dredze.
\newblock New $\mathcal{H}\infty$ bounds for the recursive least squares
algorithm exploiting input structure.
\newblock In {\em ICASSP}, pages 2017--2020, 2012.

\bibitem{DBLP:journals/ml/CrammerKD13}
Koby Crammer, Alex Kulesza, and Mark Dredze.
\newblock Adaptive regularization of weight vectors.
\newblock {\em Machine Learning}, 91(2):155--187, 2013.

\bibitem{DBLP:conf/nips/CrammerL10}
Koby Crammer and Daniel~D. Lee.
\newblock Learning via gaussian herding.
\newblock In {\em Advances in Neural Information Processing Systems 23: 24th
	Annual Conference on Neural Information Processing Systems 2010. Proceedings
	of a meeting held 6-9 December 2010, Vancouver, British Columbia, Canada.},
pages 451--459, 2010.

\bibitem{DBLP:conf/icassp/CrammerL12}
Koby Crammer and Daniel~D. Lee.
\newblock Online discriminative learning of phoneme recognition via collections
of generalized linear models.
\newblock In {\em 2012 {IEEE} International Conference on Acoustics, Speech and
	Signal Processing, {ICASSP} 2012, Kyoto, Japan, March 25-30, 2012}, pages
1961--1964, 2012.

\bibitem{DBLP:conf/colt/CrammerS03}
Koby Crammer and Yoram Singer.
\newblock Learning algorithm for enclosing points in bregmanian spheres.
\newblock In {\em Computational Learning Theory and Kernel Machines, 16th
	Annual Conference on Computational Learning Theory and 7th Kernel Workshop,
	COLT/Kernel 2003, Washington, DC, USA, August 24-27, 2003, Proceedings},
pages 388--402, 2003.

\bibitem{DBLP:conf/icml/CrammerTP08}
Koby Crammer, Partha~Pratim Talukdar, and Fernando~C. Pereira.
\newblock A rate-distortion one-class model and its applications to clustering.
\newblock In {\em Machine Learning, Proceedings of the Twenty-Fifth
	International Conference {(ICML} 2008), Helsinki, Finland, June 5-9, 2008},
pages 184--191, 2008.

\bibitem{dredze-08b}
Mark Dredze and Koby Crammer.
\newblock Active learning with confidence.
\newblock In {\em Association for Computational Linguistics}, 2008.

\bibitem{dredze-ml-09}
Mark Dredze, Alex Kulesza, and Koby Crammer.
\newblock Multi-domain learning by confidence-weighted parameter combination.
\newblock {\em Machine Learning}, 79(1-2):123--149, 2010.

\bibitem{DBLP:journals/jmlr/HaimovitchCM12}
Yoav Haimovitch, Koby Crammer, and Shie Mannor.
\newblock More is better: Large scale partially-supervised sentiment
classication.
\newblock In {\em Proceedings of the 4th Asian Conference on Machine Learning,
	{ACML} 2012, Singapore, Singapore, November 4-6, 2012}, pages 175--190, 2012.

\bibitem{DBLP:journals/bmcbi/LiuCPR08}
Qian Liu, Koby Crammer, Fernando C.~N. Pereira, and David~S. Roos.
\newblock Reranking candidate gene models with cross-species comparison for
improved gene prediction.
\newblock {\em {BMC} Bioinformatics}, 9, 2008.

\bibitem{DBLP:journals/jmlr/LivniCG12}
Roi Livni, Koby Crammer, and Amir Globerson.
\newblock A simple geometric interpretation of {SVM} using stochastic
adversaries.
\newblock In {\em Proceedings of the Fifteenth International Conference on
	Artificial Intelligence and Statistics, {AISTATS} 2012, La Palma, Canary
	Islands, April 21-23, 2012}, pages 722--730, 2012.

\bibitem{mcdonald-04}
Ryan McDonald, Koby Crammer, and Fernando Pereira.
\newblock Large margin online learning algorithms for scalable structured
classification.
\newblock In {\em NIPS Workshop on Structured Outputs}, 2004.

\bibitem{DBLP:conf/naacl/McDonaldCP05}
Ryan~T. McDonald, Koby Crammer, and Fernando Pereira.
\newblock Flexible text segmentation with structured multilabel classification.
\newblock In {\em {HLT/EMNLP} 2005, Human Language Technology Conference and
	Conference on Empirical Methods in Natural Language Processing, Proceedings
	of the Conference, 6-8 October 2005, Vancouver, British Columbia, Canada},
2005.

\bibitem{mejer2010confidence}
Avihai Mejer and Koby Crammer.
\newblock Confidence in structured-prediction using confidence-weighted models.
\newblock In {\em Proceedings of the 2010 conference on empirical methods in
	natural language processing}, pages 971--981. Association for Computational
Linguistics, 2010.

\bibitem{mejer2012training}
Avihai Mejer and Koby Crammer.
\newblock Training dependency parser using light feedback.
\newblock In {\em Proceedings of the 2012 Conference of the North American
	Chapter of the Association for Computational Linguistics: Human Language
	Technologies}, pages 488--497. Association for Computational Linguistics,
2012.

\bibitem{MoroshkoC12}
Edward Moroshko and Koby Crammer.
\newblock Weighted last-step min-max algorithm with improved sub-logarithmic
regret.
\newblock In {\em ALT}, pages 245--259, 2012.

\bibitem{DBLP:conf/pkdd/OrbachC12}
Matan Orbach and Koby Crammer.
\newblock Graph-based transduction with confidence.
\newblock In {\em Machine Learning and Knowledge Discovery in Databases -
	European Conference, {ECML} {PKDD} 2012, Bristol, UK, September 24-28, 2012.
	Proceedings, Part {II}}, pages 323--338, 2012.

\bibitem{OrbachCr12ieeei}
Matan Orbach and Koby Crammer.
\newblock Transductive phoneme classification using local scaling and
confidence.
\newblock In {\em IEEEI}, 2012.

\bibitem{DBLP:conf/ijcai/VatashskyC13}
Ben~Zion Vatashsky and Koby Crammer.
\newblock Multi class learning with individual sparsity.
\newblock In {\em {IJCAI} 2013, Proceedings of the 23rd International Joint
	Conference on Artificial Intelligence, Beijing, China, August 3-9, 2013},
2013.

\bibitem{DBLP:conf/icml/WangCV10}
Zhuang Wang, Koby Crammer, and Slobodan Vucetic.
\newblock Multi-class pegasos on a budget.
\newblock In {\em Proceedings of the 27th International Conference on Machine
	Learning (ICML-10), June 21-24, 2010, Haifa, Israel}, pages 1143--1150, 2010.

\bibitem{DBLP:journals/jmlr/WangCV12}
Zhuang Wang, Koby Crammer, and Slobodan Vucetic.
\newblock Breaking the curse of kernelization: budgeted stochastic gradient
descent for large-scale {SVM} training.
\newblock {\em Journal of Machine Learning Research}, 13:3103--3131, 2012.

\end{thebibliography}

\end{small}
%\bibliography{LowerSaxonyVorab_SiggCrammer}
% \bibliography{/home/stephan/Daten/Arbeit/Veroeffentlichungen/Literatur/Literatur_080128_v2_STS}


\label{maxSeitenzahl}


\end{document}
