\documentclass[12pt]{article}
\usepackage{setspace}

\usepackage{ucs}
\usepackage[utf8x]{inputenc}
%\usepackage{ngerman}
\usepackage{a4wide}
\usepackage{amsmath}
\usepackage{amsthm}
\usepackage{color}
\definecolor{light}{gray}{0.85}
\definecolor{heavy}{gray}{0.30}
\usepackage{amsfonts}
\usepackage{dsfont}
\usepackage{amssymb}
\usepackage{setspace}% used for table notes
\usepackage{makeidx}
\usepackage[pdftex]{graphicx}
\usepackage{multibib}
% \usepackage{subfigure}
\usepackage{subfig}
\usepackage{eurosym}
\newcites{all,own,related}%
         {\normalsize{References (other authors)},%
          \normalsize{References (applicant)},
          \normalsize{ }\vspace{-.8cm}}
          
\usepackage{color}
          
\newcommand{\todoBase}{\hfill \notiz{TODO} $\square$}
\newcommand{\projektname}{\texttt{Sense box}}

\newcommand{\notiz}[1]{\marginpar{\tiny \framebox{\begin{minipage}{2.1cm} #1 \end{minipage}}}}
\newcommand{\niklas}[1]{\marginpar{\footnotesize\notiz{Niklas:}\\\tiny \framebox{\begin{minipage}{2.1cm} #1 \end{minipage}}}}
\newcommand{\dawud}[1]{\marginpar{\footnotesize\notiz{Dawud:}\\\tiny \framebox{\begin{minipage}{2.1cm} #1 \end{minipage}}}}
\newcommand{\stephan}[1]{\marginpar{\footnotesize\notiz{Stephan:}\\\tiny \framebox{\begin{minipage}{2.1cm} #1 \end{minipage}}}}
\newcommand{\todo}{\marginpar{\todoBase}}

\newcommand{\kobyc}[1]{\begin{center}\fbox{\parbox{3in}{{\textcolor{green}{K: #1}}}}\end{center}}
\newcommand{\stephans}[1]{\begin{center}\fbox{\parbox{3in}{{\textcolor{magenta}{S: #1}}}}\end{center}}


\author{\begin{minipage}[t]{7.1cm}\centering \small \VornameAntragstellerA\ \NachnameAntragstellerA\\ \small Lehrstuhl für Kommunikationstechnik\\ \small ComTec\\ \small Universität Kassel\end{minipage}
\begin{minipage}[t]{7.1cm}\centering \small \VornameAntragstellerB\ \NachnameAntragstellerB\\ \small Gruppe Verteilte und Ubiquitäre Systeme \\ \small Institut für Betriebssysteme und Rechnerverbund\\ \small TU Braunschweig\end{minipage}}

\title{\projektname:\\\notiz{Social and sentiment sensing and assisting using on-body sensors}}

\date{\small \today}
% \usepackage[T1]{fontenc}
% \newcommand{\changefont}[3]{
% \fontfamily{#1} \fontseries{#2} \fontshape{#3} \selectfont}
%\changefont{cmss}{m}{n}
\fontfamily{cmss}
\fontseries{m}
\fontshape{n}
\selectfont
%%serifenlose Schrift:
\renewcommand{\familydefault}{\sfdefault}
\usepackage{helvet}

%% Use header as in DFG template
 \usepackage{fancyhdr}

%% Redefine plain page style
\fancypagestyle{plain}{
 \fancyhf{}
 \renewcommand{\headrulewidth}{0pt}
 \fancyfoot[LE,RO]{\thepage}
 }
 \pagestyle{plain}
% \lhead{\tiny \Name, \Strasse, \Plz\ \Ort, Germany. Phone: \Telefon, Email: \Email}
\rhead{}
% \rhead{\small page \arabic{page} of \pageref{maxSeitenzahl}}
\rfoot{}
%% Linebreak after paragraph
\makeatletter
\renewcommand\paragraph{\@startsection{paragraph}{4}{\z@}%
  {-3.25ex\@plus -1ex \@minus -.2ex}%
  {1.5ex \@plus .2ex}%
  {\normalfont\normalsize\bfseries}}
\makeatother
\begin{document}
\onehalfspacing %1.5 spacing
%% enable numbering for paragraphs also:
\setcounter{secnumdepth}{5}
% \maketitle
\pagebreak
\begin{center}
\section*{Social and sentiment sensing and assisting using on-body sensors}
\begin{tabular}{cc}
  Stephan Sigg, & Koby Crammer\\
  University of Goettingen, Germany & Department of Electrical Engineering, Technion \\
 stephan.sigg@cs.uni-goettingen.de & koby@ee.technion.ac.il
\end{tabular}

\end{center}

Gefühlszustände sind eng verknüpft mit körperlicher Gesundheit. 
So steigert die Verärgerung die Gefahr von Herzerkrankungen, Stimmungen beeinflussen ein gesund- heitliches Verhalten.
Außerdem ist mentale Gesundheit mit der Regulierung von Emotionen verknüpft wohingegen Stress signifikanten Einfluss auf die Gehirnfunktionalität, Wahrnehmung und Arbeitsleistung hat.
Demgegenüber beeinträchtigt positiver Affekt das Immunsystem während Glück und Zufriedenheit gute Indikatoren für Arbeitsleistung, Gesundheit und auch privaten Erfolg sind. 

Stimmungen und Emotionen können mit Hilfe von Physiologischen Signalen erkannt werden über Sensoren wie Electromyograme (EMG, Muskelaktivität), Electrodermische Activität (EDA, Leitfähigkeit der Haut), Electrocardiogram (EKG oder ECG, Herzaktivität), Electrooculogram (EOG, Augenbewegung) oder Electroencephalographie (EEG, Gehirnaktivität).
Solche Sensoren sind jedoch nur einer gerigen Population zugänglich und haben weitere Nachteile wie ihre hohen Kosten, Gewicht, sowie begrenzte Transportfähigkeit.

Wir werden in diesem Projekt (1) Metoden entwickeln und implementieren um Emotionen von Smartphone und Body-Sensoren zu erkennen, (2) Gefühlszustände über RSSI oder RF-signalstärke zu schätzen, (3) algorithmen für die Prognose der Sentiment-Entwicklung basierend auf vorangegangenden Messwerten entwickeln, sowie (5) Verfahren zum Feedback basierend auf Stimmungen und Emotionen entwerfen. 

Das Projektergebnis ist ein Framework für das Messen und das Feedback von Stimmungen, Emotionen und Gefühlen basierend auf smartphone, Medizinischen und Umgebungssensoren. 
Zusätzlich wird im Rahmen des Projektes eine prototypische Implementierung zur Schätzung von Gefühen, Emotionen und Stimmungen aus body, smartphone und Umgebungssensoren implementiert.  

\label{maxSeitenzahl}


\end{document}
